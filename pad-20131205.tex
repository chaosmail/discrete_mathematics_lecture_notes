%%% EtherPad for Discrete Mathematics VO
%%% http://www.informatik-forum.at/showthread.php?104454-Notes-2013WS-VO_01
%%% Past pads:
%%%     * 2013-10-17: committed by patrikf
%%%     * 2013-10-18: committed by patrikf
%%%     * 2013-10-24: committed by patrikf
%%%     * 2013-10-25: committed by patrikf
%%%     * 2013-11-08: committed by neroburner
%%%     * 2013-11-21: committed by neroburner
%%%     * 2013-11-22: committed by neroburner
%%%     * 2013-11-28: committed by neroburner
%%%     * 2013-11-29: committed by neroburner

% Discrete Mathematics Lecture Notes 2013-12-05

\subsection{Euclidean algorithm}
\lecturedate[\baselineskip]{2013-12-05}
\begin{align*}
  a &= b q_0 +r_0 \\
  b &= r_0 q_1 + r_1 \\
  r_0 &= r_1 q_2 + r_2\\
  r_{k-2} &= r_{k-1}  q_k + r_k \\
  r_{k-1} &= r_{k}  q_{k+1} + 0 \\
\end{align*}

all the previous formulas: $ b > r_0 > r_1 > r_2 > \ldots > r_k > 0$. From this follows
\[
  r_k = gcd(a,b)
\]

\Proof. \\
\begin{tabular}{ll}
  1) & $r_k | r_{k-1} \implies r_k | r_{k-1} q_k + r_k = r_{k-2}$ \\
     & $\implies \ldots \implies r_k | a, r_k | b$ \\
  2) & $r | a \wedge t | b \stackrel{\text{??}}{\implies} t| r_k$ \\
     & $\implies t | a-bq_0 \implies t | r_1 \implies \ldots \implies t|r_k$
\end{tabular}


\Remark.
\begin{align*}
  r_j &= r_{j + 1} \underbrace{q_{j+2}}_{ \geq 1} + r_{j+2} \geq \underbrace{r_{j+1}}_{ > r_j+2} + r_{j+2} > 2 r_{j+2} \\
      & \forall j: r_{j+2} < \frac{r_j}{2}
\end{align*}

\Theorem.
\[
  d = gcd(a,b) \implies \exists e,f \in \mathbb{Z}: d = ae + bf
\]

\Proof.
reverse Euklidian algorithm

\begin{definition}
$R$ is a \dt{commutative ring} with $1$. \\
$(R,t)$ \dt{abelian group} $\rightarrow 0$, inverse of $a$ is $-a$\\
$(R,\cdot)$ \dt{semigroup}, if $\exists$ neutral element: $1$ \\
distributive laws\\
$R$ \dt{integral domain} $\iff \nexists a,b \in R \backslash \{0\}: a*b = 0$
\end{definition}

\textbf{Example.}
$(\mathbb{R}, +, \cdot)$, \\
$(\mathbb{Z}, + , \cdot), m \in P$

$m\not\in \mathbb{P}: m = n \cdot k : \bar{n} \cdot \bar{k} = \bar{m} = \bar{0}$

$\mathbb{Z}_6 : \bar{2} \cdot \bar{3} = \bar{0}$ \\
$\{a_0 + a_1 x + a_x x^2 \ldots a_n x^n \mid a_i \in \mathbb{Z}, n \in \mathbb{N} \}$

\begin{definition}
$R$ \dt{Euklidean ring} if $R$ integral domain and there is an Euklidean function $n$:
\[
  n : R \rightarrow \mathbb{N} \text{ s.t } \forall a,b \in R,\; b \neq 0,\; \exists q,r \in R:
\]

\begin{compactenum}
\item $a = bq + r$ with $r = 0$ or $n(r) < n(b)$
\item $\forall a,b \in R \backslash \{0\} : n(a) \leq n(ab)$
\end{compactenum}
\end{definition}

\textbf{Example.}
all natural numbers $\mathbb{Z}$, because $n(a) = |a|$

$K$ is a field, and $K[x] = 
\left(\{a_0 + a_1 x + a_2 x^2 \ldots a_n x^n \mid a_i \in K, n \in \mathbb{N} \}
,+, \cdot \right)$

$K[x]$ is Euklidean ring:
\begin{align*}
  &p(x) \in K[x] \\
  &n(p(x)) = deg(p(x))
\end{align*}

\begin{definition}
a \dt{field} has a addition and a multiplication
$(K, +, \cdot):$ 
  $(K,+)$ abelean group (associativ 0, $\forall a \exists -a$) \\
  $(K \backslash \{0\}, \cdot)$ abelean group \\
  distributive laws
\end{definition}

\textbf{Example.}
\begin{align*}
  (x^4 + 3x^3 - 3x^2 - 7x + 6) &: ( x^3 + x^2 - x + 15) = x+2 \\
  -x^4 - x^3 + x^2 - 15 x \\
  \hline 
    2x^3 - 2x^2 - 22 x + 6 \\
    -2x^3 - 2x^2 + 2 x -30 \\ 
    \hline
      -4x^2 - 20 x - 24
\end{align*}

\begin{definition}
$p>1$ is called a \dt{prime number} $\iff$ $\pm 1, \pm p$ are the only divisors.

$\mathbb{P}$ set of primes
\end{definition}

\Theorem.
\begin{align*}
  & p\in \mathbb{P}, \\
  & p|ab \implies p|a \vee p|b
\end{align*}

\Proof.

Case 1) $p|a$, already true\\
Case 2) $p \not| a$
\begin{align*}
  &\implies gcd(p,a) = 1 \\
  &\implies \exists e,f \in \mathbb{Z}: ep + fa = 1 \\
  &\implies b = b*1 = b e p + b f a \ldots \text{ multiple of p } \implies p|b
\end{align*}

\Theorem.
\[
  n \in \mathbb{N}^{+} \implies \exists p_1, \ldots ,p_r \in \mathbb{P}: n = p_1 \ldots p_r, (r \geq 0), (\prod_{i\in \varnothing} a_i = 1)
\]

\Remark.
factorisation is unique up to order of the factors

\[
  \Rightarrow n = \prod_{p \in \mathbb{P}} p^{\nu_p(n)}
\]

$\nu_p(n)$ \ldots multiplicity of $p$ in the factorisation of $n$

\[
  p^{\nu_p(n)} | n, \quad p^{\nu_p(n)+1} \not| n
\]
\begin{align*}
  gcd(a,b) &= \prod_{p \in \mathbb{P}} p ^{\min(\nu_p(a), \nu_p(b))} \\
  lcd(a,b) &= \prod_{p \in \mathbb{P}} p ^{\max(\nu_p(a), \nu_p(b))} \\
  a|b &\iff \forall p \in \mathbb{P} : \nu_p(a) \leq \nu_p(b)
\end{align*}

\Theorem.
\[
  | \mathbb{P} | = \infty
\]

\Proof.
\[
  \mathbb{P} = \{p_1, \ldots , p_r\}, N = p_1 \ldots p_r + 1 = \prod_{i=1}^r p_i^{\nu_p(N)}
\]
Contradiction!

\section{Congruence relations and residue classes}

\begin{definition}
$m \in \mathbb{N}^{+}$ modulus, 
residue class: $a+m \mathbb{Z} = \bar{a}$ 
\end{definition}

\Remark.
\[
  a\in \bar{a},\; \bar{a} = \bar{b} \iff m | a-b
\]
\begin{align*}
  a \equiv b \mod m & :\iff m | a-b \\
  a \equiv b (m)     & :\iff \bar{a} = \bar{b}
\end{align*}
\[
  \mathbb{Z}_m = \{ \bar{0},\bar{1},\bar{2}, \ldots, \overline{m-1}\} , 
  \bar{0} \cup \bar{1} \cup \bar{2} \cup  \ldots \cup \overline{m-1} = \mathbb{Z}
\]


\begin{definition}
\[
  \bar{a} + \bar{b} := \overline{a+b}
\]
\[
  \bar{a} \cdot \bar{b} := \overline{ab}
\]
\end{definition}

\Remark.
\[
  a \equiv c(m), b \equiv d(m) \rightarrow 
\]
\begin{align*}
  a+b \equiv c+d(m) \\
  ab \equiv cd (m)
\end{align*}

\Theorem.
$(\mathbb{Z}_m, + , \cdot)$ is a commutative ring with $1$

\begin{definition}
  We define the inverse element as follows: 
\[
  \bar{a} \in \mathbb{Z}_m, \bar{x} \in \mathbb{Z}_m \text{ s.t }
  \bar{x} \cdot \bar{a} = \bar{1} \implies \bar{x} = \bar{a}^{-1}
\]
\end{definition}

\textbf{Example.}
\begin{align*}
  m = 5: &\quad \bar{2}^{-1} = \bar{3} \\
  m = 6: &\quad \bar{2}\cdot\bar{3} = \bar{0} \Rightarrow \bar{x} \cdot \bar{2} \cdot \bar{3} = \bar{0} \quad \bar{x} \cdot \bar{2} \neq \bar{1} \Rightarrow \nexists \bar{2}^{-1}
\end{align*}

\Theorem.
$\exists \bar{a}^{-1} \text{ in } \mathbb{Z}_m \iff gcd(a,m) = 1$ ($a,m$ are co-prime)

\Proof.

\ProofForward.
  $\bar{a}\cdot \bar{x} = \bar{1} \Rightarrow \exists k\in Z : 
  ax = 1 + km \Rightarrow ax - km = 1$

  $d = gcd(a,m) \Rightarrow d \mid ax - km \Rightarrow d = 1$

\ProofBackward.
  $gcd(a,m) = 1 \Rightarrow \exists e,f : ae + mf = 1$ \\
  $\Rightarrow ae = 1 + (-f)m \Rightarrow \bar{a} \cdot \bar{e} = \bar{1} \Rightarrow \bar{e} = \bar{a}^{-1}$

\begin{definition}
\[
  \mathbb{Z}_m^{*} = \{ \bar{a} \in \mathbb{Z}_m \mid gcd(a,m) = 1 \}
\]
set of prime redisue classes mod m
\end{definition}

\textbf{Example.}
\[
  \mathbb{Z}_5^{*} = \{ \bar{1}, \bar{2}, \bar{3}, \bar{4} \}, \mathbb{Z}_6^{*} = \{ \bar{1}, \bar{5} \}
\]

\textbf{Example.}
$ m = 17$, find $\overline{13}^{-1}$: $ 13 x \equiv 1 (17)$ $x \equiv 4 (17)$
\begin{align*}
  &17 = 13 * 1 +4\\
   &13 = 4*3 +1 \iff 1 = 13-4*3 = 13-(17-13)*3 = 17*3+4*13 \\
   & \Rightarrow x \equiv 4 (17)
\end{align*}

\textbf{Example.}
\begin{align*}
  3b &\equiv 3c (5) | *2 \\
  b &\equiv c(5)
\end{align*}

\textbf{Example.}
\begin{align*}
  3b &\equiv 3c (6) \\
  3b &= 3c + k*6 | :3 \\
  b &= c + k*2 \\
  b &\equiv c(5)
\end{align*}

\textbf{Example.}
\begin{align*}
  ab &\equiv ac (am) \Rightarrow b \equiv  c (m) \\
  ab &\equiv ac (m) \Rightarrow b \equiv c (m)
    \text{ if } ax = 1 (m) \text{ has a solution } \iff gcd(a,m) = 1
\end{align*}

\section{Systems of congruences}

\Theorem.
$m = m_1 m_2$, $gcd(m_1, m_2) = 1$. Then $x \equiv y (m) \iff \begin{cases} x \equiv y (m_1) \\ x \equiv y (m2) \end{cases}$

\ProofForward.
\[
  x \equiv y (m) \iff x = y + km \implies x = y + k m_1 m_2 \implies
  \begin{cases}
    x \equiv y (m_1) \\
    x \equiv y (m_2) 
  \end{cases}
\]

\ProofBackward.
\[
  \begin{cases}
    x \equiv y (m_1) \\
    x \equiv y (m_2) 
  \end{cases}
\]
\begin{align*}
  & \implies \exists l: x-y = l* m_1 \equiv 0 (m_2)\\
  & l \equiv 0 (m_2) \implies l = l' * m_2 
    \implies x-y = l'* m_1 m_2
    \implies x \equiv y (m)
\end{align*}

\textbf{Corollary.}
\[
  m = \prod_{i=1}^r m_i,\; gcd(m_i, m_j) = 1 \forall i \neq j
\]

Then $x \equiv y (m) \iff \forall i = 1, \ldots, r : x \equiv y (m_i)$

\subsection{Chinese remainder theorem}
We are faced with a system of congruence equations
\[
  x \equiv a_i (m_i) \quad 1 \leq i \leq r (*)
\]
$i \neq j \implies \gcd(m_i, m_j) = 1$

$\implies$ $(*)$ has a unique solutiion mod $m = \prod_{i = 1}^r m_i$.

and this is $x \equiv \sum_{j=1}^r \frac {m}{m_j} * b_j * a_j (m)$

where $b_j = \left( \frac{m}{m_j} \right) ^{-1} (m_j)$
