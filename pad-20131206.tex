%%% EtherPad for Discrete Mathematics VO
%%% http://www.informatik-forum.at/showthread.php?104454-Notes-2013WS-VO_01
%%% Past pads:
%%%     * 2013-10-17: committed by patrikf
%%%     * 2013-10-18: committed by patrikf
%%%     * 2013-10-24: committed by patrikf
%%%     * 2013-10-25: committed by patrikf
%%%     * 2013-11-08: committed by neroburner
%%%     * 2013-11-21: committed by neroburner
%%%     * 2013-11-22: committed by neroburner
%%%     * 2013-11-28: committed by neroburner
%%%     * 2013-11-29: committed by neroburner
%%%     * 2013-12-05: committed by neroburner

% Discrete Mathematics Lecture Notes 2013-12-06

\lecturedate[\baselineskip]{2013-12-05}

\textbf{rembember from last time}
\[
  x \equiv a_i (m_i), 1 \leq i \leq r (*)
\]
$\implies$ (*) has unique solution mod $m = \prod_{i = 1}^r m_i$. 

\[
  x \equiv \sum_{j=1}^r \frac {m}{m_j} * b_j * a_j (m) \text{ where }
  b_j = \left( \frac{m}{m_j} \right) ^{-1} \mod m_j
\]

\textbf{Example.}
System of two congruent relations.
\begin{align*}
  3x \equiv 2 (5) \\
  2x \equiv 7 (11)
\end{align*}

The previous are equivalent to
\begin{align*}
  x \equiv 4 (5) \\
  x \equiv 9 (11)
\end{align*}

\begin{align*}
  &m_1 = 4, m_2 = 11, m = 55 \\
  &a_1 = 4, a_2 = 9 \\
  \hline
  &b_1 * 11 = 1 (5)  \implies b_1 = 1 \\
  &b_2 *5 = 1 (11) \implies b_2 = 9 \\
  \hline
  & x \equiv 11 * 1 * 4 + 5 *9 *9 \mod 55\\
  & x \equiv 449 mod 55 \\
  & x \equiv 9 (55)
\end{align*}

\Proof.
1) x is a solution: $m_i, i = 1,2, \ldots, r$, are pairwise coprime 
\[
  \implies gcd(\frac{m}{m_j}) = 1 \implies \exists b_j
\]

\begin{align*}
  \frac{m}{m_j} \equiv 0 (m_j) \forall i \neq j 
    \implies \sum_{j=1}^{r}\frac{m}{m_j} b_j * a_j \equiv 
    \frac{m}{m_j} \underbrace{b_i}_{\left(\frac{m}{m_j}\right)^{-1} \mod m_i} a_i (m_i) \\
    \equiv a_i (m_i)
\end{align*}

2) x is unique mod m:
\begin{align*}
  x \equiv a_i(m_i) y \equiv a_i (m_i) \forall i = 1,\ldots, r \\
  \implies x \equiv y (m_i) \forall i \implies x\equiv y (m)
\end{align*}

\section{Eurler-Fermat theorem and RSA-Algorithm}

\Theorem.
$(\mathbb{Z}_m^{*}, \cdot)$ is a group.
\begin{align*}
  &|\mathbb{Z}_m| = m, \\
  &|\mathbb{Z}_m^{*}| = \varphi(m) \\
  &\varphi \ldots \text{ Euler's totient function}
\end{align*}

E.g.: $\varphi(5) = 4, \varphi(6) = 2$

\begin{align*}
  &m \in \mathbb{P}: q(m) = m-1 \\
  &m = p^e, p \in \mathbb{P}, e \geq 1 \\
  &\bar{a}\in \mathbb{Z}_m \implies gcd(a,p^e) = \begin{cases}
    1 \text{ if $\bar{a} \in \mathbb{Z}_m^{*}$} \\
    p^f \text{ $1 \leq f \leq e$} 
  \end{cases} \\
  &\bar{a}\in \mathbb{Z}_m^{*} \Leftrightarrow p X a \\
  & p^e - p^{e-1} = \varphi(p^e) = p^{e-1}(p-1) \\
  & m = m\left(1-\frac{1}{p}\right) 
\end{align*}

\Theorem.
\[
  m = p_1^{e_1} p_2^{e_2} \ldots p_r^{e_r} \implies
  \varphi(m) = m \left( 1-\frac{1}{p_1} \right) \left( 1-\frac{1}{p_2} \right) \ldots \left( 1-\frac{1}{p_r} \right)
\]

\Proof.
Let $r=2: M = \{1,2, \ldots, m\},$ 
\begin{align*}
  A = \{ n \in M \mid p_1|n\} = \{p_1, 2p_1, 3_p1, \ldots m-p_1, m\} \\
  B = \{ n \in M \mid p_2|n\} = \{p_2, 2p_2, 3_p2, \ldots m-p_2, m\} \\
\end{align*}

\begin{align*}
  |M| = m, |A| = \frac{m}{p_1}, |B| = \frac{m}{p_2}, 
  |A \cap B| = \frac{m}{p_1p_2}\\
  \varphi(m) = | M \backslash(A\cup B) | = |M| - |A| - |B| ü | A cap B| 
    = \left(m (1- \frac{1}{p_1} \right) \left(m (1- \frac{1}{p_2} \right)
\end{align*}

\Theorem.
Therem of Euler-Fermat:
\[
  gcd(a,m) = 1 \implies a ^{\varphi(m)} \equiv 1 (m)
\]
In particular 
\[
  p \in \mathbb{P}, p \not\mid a \implies a^{p-1} \equiv 1 (p)
\]
(Fermat's little theorem)

\Proof.
\[
  \mathbb{Z}_m^{*} = \{\bar{a}_1, \bar{a}_2, \ldots, \bar{a}_k \}, k = \varphi(m)
\]
\begin{align*}
  \implies \mathbb{Z}_m^{*} \{\bar{a}\bar{a}_1, \bar{a}\bar{a}_2, \ldots, \bar{a}\bar{a}_k \}, 
    \text{ since } \bar{a}\bar{a}_i = \bar{a}\bar{a}_j 
    \implies \bar{a}_i = bar{a}_j \implies i = j
    \implies \underbrace{\bar{a}_1, \bar{a}_2, \ldots, \bar{a}_k}_
        {\in \mathbb{Z}_m^{*}}
      = \bar{a}^k * \underbrace{\bar{a}_1, \bar{a}_2, \ldots, \bar{a}_k}_
        {\in \mathbb{Z}_m^{*}}
      \implies \bar{a}^k = \bar{a}^{\varphi(m)} = \bar{1} 
\end{align*}

\Theorem.
\[
  p, \in \mathbb{P}, p,q odd, p \neq q
\]
\begin{align*}
  m= pq, v = lcm(p-1, q-1) \\
  \implies \forall a,k \in \mathbb{Z}: a ^{kv + 1} \equiv a(m)
\end{align*}

\Proof.
To show: $pq \mid a ^{kv +1} - a$
\begin{align*}
  &\Leftrightarrow p \mid a ^{kv +1} - a \land q \mid a ^{kv +1} - a
  &\text{Case 1: } p \mid a \\
  &\text{Case 2: } p\not\mid a: 
    a^{p-1} \equiv 1 (p) \implies a^{kv} \equiv 1(p)
\end{align*}

\subsection{RSA-Algorithm}
\[
  m = pq, gcd(e,v) = 1 \implies \exists d : d *e \equiv 1 (v)
\]
$v = lcm(p-1, q-1)$

We have a message $a_1, a_2, a_3, \ldots$ where $0 \leq a_i < m$

Encryption: 
\[
  E(a_j) = b_j := a_j^e \mod m
\]

Decryption: 
\[
  D(b_j) = a_j := b_j^d \mod m
\]

Check if this really works:
\begin{align*}
  b_j^d &\equiv (a_j^e)^d \mod m \\
    & \equiv a_j^{e*d} \mod m \\
    & \equiv a_j^{kv+1} (m) \\
    & \equiv a_j (m) \\
\end{align*}

We can provide a public key $(m,e)$ and The private kay $d$.

If we want to compute the private key if we only know the public key the hard part is the factorisation $m = p*q$. This factorisation needs time exponential in the number of digits of $m$.

\textbf{e-signature: } 
$(e_j,d_j)$,

$e_j$ is public $\hat{=} E_j, D_j$

User i sends to user j:
\[
  E_j(D_i(x)) = x^{d_i e_j} (m)
\]
User j checks with his private key:
\[
  D_j (E_J(D_i(x))) = D_j(x)
\]
then $E_i$ afterwards:
\[
  E_i(D_i(x)) = x
\]

\textbf{Caveat. (Warning)}
$E(x)$ may have many fixed points. 
If the order is low then it is possible to find the $x$. 

\textbf{Recall.}
If $G$ is a group and $x\in G$:
\[
  org_G(x) = min\{ i\in \mathbb{N}^{+} \mid x^i = e \}
\]
\[
  e^1 = e \implies ord_G(e) = 1, a \neq e \implies ord_G(a) > 1
\]

\textbf{Cyclic group.}
the group generated by $x$:
\[
  \{ e, x, x^2, x^3, \ldots \} = \left\langle x \right\rangle
\]

G is finite:
\begin{align*}
  U \leq  G \implies |U| \mid |G|, 
  & \left\langle x \right\rangle = \{ e, x, x^2, x^3, \ldots x^{ord_G(x) -1}\}
    &\implies |\left\langle x \right\rangle| = ord_G(x) \\
    &\implies ord_G(x) | \underbrace{|G|}_{text{order of the group}} \\
\end{align*}

G is cyclic
\begin{align*}
  & \Leftrightarrow \exists x \in G: \left\langle x \right\rangle = G \\
  & \Leftrightarrow \exists x \in G: ord_G(x) = |G|
\end{align*}
%  \left\langle    \right\rangle

\begin{definition}
  \[
    \bar{a} \in \mathbb{Z}_m^{*} \text{ s.t. } \left\langle \bar{x} \right\rangle = \mathbb{Z}_m^{*} 
\]
is called \dt{primitive root mod m}

$\bar{a}$ primitive root mod m 
$\implies \mathbb{Z}_m^{*} = \{\bar{a}, \bar{a}^2, \bar{a}^3, \ldots, \bar{a}^{\varphi{m)}} \}$
\end{definition}

\def\GroupGenBy#1{\left\langle #1 \right\rangle}

\textbf{Example.}
\begin{align*}
  \mathbb{Z}_2^{*} = \{\bar{1}\} = \GroupGenBy{\bar{1}} \\
  \mathbb{Z}_3^{*} = \{\bar{1}, \bar{2}\} = \GroupGenBy{\bar{2}} 
    = \{\bar{2}, \bar{2}^2 \}\\
  \mathbb{Z}_4^{*} = \{\bar{1}, \bar{3}\} = \GroupGenBy{\bar{3}} \\
  \mathbb{Z}_5^{*} = \{\bar{1}, \bar{2}, \bar{3}, \bar{4}\} 
    = \GroupGenBy{\bar{2}} = \GroupGenBy{\bar{3}} \neq \GroupGenBy{\bar{4}}
    = \{\bar{1}, \bar{4}\} \\
\end{align*}

\TODO{finish example}

\Theorem.
$\mathbb{Z}_m^{*}$ cyclic $\Leftrightarrow$ $\exists$ prim. root mod m 
\begin{align*}
  \Leftrightarrow 
    m \in \{2,4\} \cup \{ p^e mid p \in \mathbb{P} \backslash \{2\}, e \geq 1 \} \cup \{ 2p^e mid p \in \mathbb{P} \backslash \{2\}, e \geq 1 \}
\end{align*}

\Proof.
Let $g$ be a primitive rot mod $p$. Then $g$ or $g+p$ is a primitive root mod $p^2$.


\textbf{Claim.}
$g^{p-1} \not{\equiv} 1 (p^2)$ or $(g+p)^{p-1} \not\equiv 1 (p^2)$

\textbf{Note.}
$\varphi(p^2} = p(p-1)$, assume that $p-1 = k*l (k, l < p-1)$

$ord_{\mathbb{Z_p^2}}^{*}(g) = p*l \implies g^{pl} \equiv 1 (p^2)$



