%%% EtherPad for Discrete Mathematics VO
%%% http://www.informatik-forum.at/showthread.php?104454-Notes-2013WS-VO_01
%%% Past pads:
%%%     * 2013-10-17: committed by patrikf
%%%     * 2013-10-18: committed by patrikf
%%%     * 2013-10-24: committed by patrikf
%%%     * 2013-10-25: committed by patrikf
%%%     * 2013-11-08: committed by neroburner
%%%     * 2013-11-21: committed by neroburner

% Discrete Mathematics Lecture Notes 2013-11-22

Binary tree of even size has a number of internal nodes $a_n$ = \frac{1}{n+1) {2n \choose n}$
$A(z) = 1 + zA(z)^2$

if we look at the set of Binary trees $B$ the tree is either a leaf with size 0, or a node with two binary trees

\textbf{Examples.}

\textbf{Example 1.}
We have a bin with red, blue and yellow balls. 
We have 2 or 3 red balls, at least one blue ball, not more than one yellow ball.
We now ask for ne number of combinations with n balls, whith the given constraints. 

Assume we introduce a variable for each color, $r,b,y$. And we count by the exponent. So our function is:
\[
    1*r^2 + r^3,  1+y, b + b^2 + \ldots = \frac{b}{1-b}
\]

we now have 2 configurations:
\[
    A(z) = \sum a_n z^n
    B(z) = \sum b_n z^n
    \implies \sum_{k=0}^{n} a_k b_{n-k}
\]

This leads to or formula
\[
    (r^2  z^2+ r^3 z^3) (1+zy) (\frac{bz}{1-bz}) 
    = \sum a_{lmkn} r^l b^m y^k z^n
\]

The coeffizient $a_{lmkn}$ is the number of configurations with l red, m blue, k yellow and total number equal to n balls

We are not interested in the number of blue,... balls but the total number of used balls, so we can set $r=b=y=1$ and we can simplify the formula to:
\[
    (z^2 + z^3)(1+z) \frac{z}{1-z} 
    \implies \text{# comb} 
    = [z^n] \frac{z^3(1+z)^2}{1-z}
    = [z^{n-3}] \frac{1+2z +z^2}{1-z}
    = [z^{n-3}] \frac{1}{1-z} + 2[z^{n-4} \frac{1}{1-z} + [n^{n-5}] \frac{1}{1-z} 
    = 4
\]

\textbf{Example.} 
Number of combination w/o repetition of a set $M = \{1,2, \ldots ,N\}$. We now want to know the number of combinations of size $k = {N \choose k}.

\[
    a_1, \ldots , a_n \hat{=} different balls \hat{=} elemets of M
\]

\[
    a_1^0, a_1^1 
    (1+a_1)(1+a_2)(1+a_3) \ldots (1+a_N)
\]

We now need to consider the size of the set
\[
    (1+x)(1+x)(1+x) \ldots (1+x)^N = \sum {N \choose k} x^k
\]

If we allow repetition: 
\[
    \Pi _{i=1}^N (1+a_i + a_i^2 + \ldots) = \Pi _{i=1}^N \frac{1}{1-a_i} \quad a_i = x
\]

\[
    f(x) = \Pi _{i=1}^N \frac{1}{1-x} = frac{1}{(1-x)^N} 
    = (1-x)^{-N} 
    = \sum _{k \geq 0} {-N \choose k} (-1)^k x^k
    = \sum _{k \geq 0} {N + k-1 \choose k} x^k
\]

\subsection{Combinatorial construction}
$mathbb{A}$ is a combinatrial class, which is a set of objects. 
Furthermore there is a size-function $w: mathbb{A} \rightarrow \mathbb{N}$. 

\[
a_n = \left(\text{number of objects x} \in mathbb{A} \text{such that w(x) = n\right) < \infty \forall n \in mathbb{N}
\]

generating function of $(mathbb{A}, w) 
\[
    A(z) = \sum_{n\geq 0 } a_n z^n
\]

We now create combinatorial classes from combinatorial classes.
(mathbb{A}, w_A) , (mathbb{B}, w_b)

1) combinatorial sum $mathbb{A} + mathbb{B}$:
    we assume that $mathbb{A} \cap mathbb{B} = \varnothing$
    \[
        mathbb{A} + mathbb{B} = (mathbb{A} \cup mathbb{B} , w), 
        w(x) = \choice
            {w_A(x) \quad x \in mathbb{A}}
            {w_B(x) \quad x \in mathbb{B}}
        c_n a_n + b_ \implies C(z) = A(z) + B(z)
    \]
2) 
    \begin{aligh*}
        $mathbb{A} \times mathbb{B} 
        := (mathbb{A} \times mathbb{B}, w) \\
        \text{with } w((x,y)) = w_a(x) + w_b(y) x\in mathbb{A}, y\in mathbb{B}\\
        c_n = \sum_{k=0}^n a_kb_{n-k}\\
        C(z) = A(z)B(z)
    \end{align*}

3)
    \begin{align*}
        seq(mathbb{A}) = \{ (x_1, x_2, \ldots x_k) \mid k \in mathbb{N} x_i \in \mathbb{A}\}\\
        k = 0 \hat{0} \epsilon \ldots \text{empty sequence}\\
        w((x_1, x_2, \ldots x_k) ) = \sum_{i=1}}^{k} w_A(x_i) \\
        \mathbb{C} = seq(\mathbb{A} = \{\epsilon\} \cup \mathbb{A} \cup \mathbb{A}\times\mathbb{A} \cup \ldots \\
        C(z) = 1 + A(z) + A(z)^2 + \ldots = \frac{1}{1-A(z)} \\
    \end{align*}
    
    Assumtion $a_0 = 0$

\textbf{Example.}
Integer paritions:

Decomposition of an integer into a sum of smaller integers ($5 = 3+1+1= 1+3+1 = 2+2+1 = 1+1+1+1+1$). The order does NOT matter ($5= 4+1 = 3 + 2 = 3+1+1 = 2+2+1 = 2*1+1+1 = 1+1+1+1+1$). 

If the order matters, then we speak of \dt{compositions} of an integer

\begin{align*}
    \mathbb{C} \ldots \text{compositions of an integer }\in \mathbb{N}^{+} \\
    \mathbb{C} = \text{seq}(\mathbb{A}) \\
    && \mathbb{A} = \mathbb{N}^{+}, w_A(x) = x\\
    && \underline{x} = (x_1, x_2, \ldots x_k) \in \mathbb{C}
    &&& \underline{x} \hat{=} x_1 + x_2 + \ldots x_k
    &&& w(\underline{x}) = x_1 + x_2 + \ldots x_k
    \mathbb{A}=seq( \{\dot \})\backslash \{\epsilon\}) \\
    A(z) = \frac{1}{1-z} - 1 = \frac{z}{1-z} \\
    C(z) = \frac{1}{1-A(z)} = \frac{1}{1-\frac{z}{1-z} }
        = \frac{1-z}{1-2z} 
        = 1 + \sum_{n\geq 1} 2^{n-1} z^n \\
    P(z) = \Pi_{i\geq 1} \frac{1}{1-x^i} \\
        &&\quad \mathbb{P} = seq(\{1\}) \times seq(\{2\}) \times \ldots \\
\end{align*}

\text{Example.}
Strings composed of $\cdot, --$. 
$\cdot$ has length 1 and $--$ has length 2.
We want to compute the number of strings of length $n$
\begin{align*}
    \mathbb{A} = \{\cdot\} \leftrightarrow z\\
    \mathbb{V} = \{\ -- \} \hat{=} \mathbb{A} \times \mathbb{A} \leftrightarrow z^2 \\

\textbf{Example.}
Binary trees, as explained before
\begin{align*}
    \mathbb{B} = \{ \Box\} + \{ \circ\} \times \mathbb{B} \times \mathbb{B}$
    B(z) = 1 + z * B(z) * B(z)
\end{align*}

\subsection{Labelled constructions}
\textbf{Example.}
permutations, $a_k = n!$

$A(z) = \sum_{n\geq 0} n! z^n$

cyclic permutations: $(p_1 p_2 \ldots p_n) \hat{=} (p_3 p_4 \ldots p_n p_1 p_2)$

so the number of cyclic permutations is $(n-1)!$

With the generating function: 
\begin{align*}
    B(z) = \sum (n-1)! z^n
\end{align*}

\begin{definition}
We define a \dt{labelld structure} $\mathbb{A} in the following way: Each object of size n is composed of n atomc objects. The atoms are numbered with numbers from $1$ to $n$. 

$A \leftrightarrow \text{generating function} \hat{A}(z) = \sum_{n\geq 0} a_n \frac{z^n}{n!}$

\begin{align*}
    \mathbb{C} = \mathbb{A}+\mathbb{B}: \\
        & w(x) = \choice{
                w_a(x) \quad x\in \mathbb{A}
                w_B(x) \quad x\in \mathbb{B}}
    \hat{C}(z) = \sum_{n\geq 0} (a_n+b_n) \frac{z^n}{n!}
        = \hat{A}(z) + \hat{B}(z)
\end{definition}

\TODO{complete the definition from notes, $\mathbb{A}*\mathbb{B}$ etc}
}

\begin{align*}
    c_n = \sum_{k=0}^{n} {n \choose k} a_k b_{n-k}
\end{align*}

\TODO{figure of $k$ elements $\in \mathbb{A}$}

\begin{align*}
    \frac{c_n}{n!} &= \sum{k=0}^{n} \frac{1}{k!(n-k)!} a_k b_{n-k}\\
    &= \sum{k=0}^{n} \frac{a_k}{k!} \frac{b_{n-k}}{(n-k)!} \\
    [z^n] \hat{A}(z)\hat{B}(z) 
        \quad [z^k] \hat{A}(z) 
        \quad [z^{n-k}] \hat{B}(z) \\
    \implies \hat{C}(z) = \hat{A}(z) \hat{B}(z) \\
\end{align*}

\begin{align*}
    seq(\mathbb{A}) = \{\epsilon\} \times \mathbb{A} \times \mathbb{A}*\mathbb{A} \times \ldots \\
    \hat{C}(z) = \frac{1}{1-\hat{A}(z)} \
    set(\mathbb{A}) "\tilde{=}" \{\varnothing\} 
        \times \mathbb{A} 
        \times \frac{1}{2}\mathbb{A} * \mathbb{A} + \frac{1}{3!}\mathbb{A}*\mathbb{A}*\mathbb{A} \\
        \implies \hat{C}(z) = e^{\hat{A}(z)
\end{align*}

sideremark: in the unlabelled case: 
\begin{align*}
    set(\mathbb{A}): C(z) = \exp{A(z) - \frac{A(z^2)}{2} + \frac{A(z^3)}{3} + \ldots }
\end{align*}

cycle of \mathbb{A}
\begin{align*}
    cyc(\mathbb{A}) \tilde{=} \mathbb{A} + \frac{1}{2}\mathbb{A} * A + \frac{1}{3}\mathbb{A}*\mathbb{A} + \mathbb{A} + \mathbb{A} + \ldots \frac{1}{k}\mathbb{A}^k \\
    \hat{C}(z) = \log\frac{1}{1-\hat{A}(z)}
\end{align*}

\textbf{Example.}
Permutations. 
\begin{align*}
    \mathbb{P} &= set(cyc(\{ \circ \} )) \\
    \hat{p}(z) &= e^{log \frac{1}{1-z}} = \frac{1}{1-z} \\
        &= \sum_{n\geq 0} n! * \frac{z^n}{n!}
\end{align*}



