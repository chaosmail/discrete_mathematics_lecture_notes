%%% EtherPad for Discrete Mathematics VO
%%% http://www.informatik-forum.at/showthread.php?104454-Notes-2013WS-VO_01
%%% Past pads:
%%%     * 2013-10-17: committed by patrikf
%%%     * 2013-10-18: committed by patrikf
%%%     * 2013-10-24: committed by patrikf
%%%     * 2013-10-25: committed by patrikf
%%%     * 2013-11-08: committed by neroburner
%%%     * 2013-11-21: committed by neroburner
%%%     * 2013-11-22: committed by neroburner
%%%     * 2013-11-28: committed by neroburner

% Discrete Mathematics Lecture Notes 2013-11-29

\textbf{Example}.
\lecturedate[\baselineskip]{2013-11-29}
Sets $A_1, A_2, \ldots A_m \subseteq M$. We take the powerset of set of indices $(2^{\{1,2, \ldots , m\}}, \supseteq)$. 
Note that the Möbius Function only depends on the structure of the order. 

\[
  \mu (A,B) = \begin{cases} 
    (-1)^{|B| - |A|} &, A\supseteq B\\ 
    0                & \text{, otherwise}
  \end{cases}
\]

\begin{gather*}
  I \subseteq \{ 1,2, \ldots, m\}, \\
  f(I) = \left| \bigcap_{i\in I} A_i \cap 
    \bigcap_{j \in \{1,\ldots ,m\} \backslash I} \bar{A}_j \right| \\
  S_f(I) = \sum_{J \supseteq I} f(J) = \left| \bigcap_{i\in I} A_i  \right| \\
  \left[ S_f(x) = \sum_{z\leq x} f(z) \right]
\end{gather*}

\begin{align*}
  f(I) &= \sum_{J \supseteq I} S_f(I) \mu(J,I) \\
    &= \sum_{J \supseteq I} (-1)^{|J| - |I|} * \left| \bigcap_{i\in I} A_i \right| \\
  f(\varnothing) &= \left| \bigcap_{j \in \{1,\ldots ,m\}} \bar{A}_j \right| \\
    &= \sum_{j \in \{1,\ldots ,m\}} (-1)^{|J|} \left| \bigcap_{j \in J} A_j \right|
\end{align*}

\textbf{Example.}
Poset $(\mathbb{N}^{+}, \mid)$.
We can say if n is a product $n = p_1^{g_1} \ldots p_r^{g_r}$ and $m$ is the product $m = p_1^{f_1} \ldots p_r^{f_r}$. This implies 
\[
  m|n \iff \forall f_i \leq g_i
\]

The number theoretical number order $(\mathbb{N}^{+}, \mid)$ is isomorphic to $(\mathbb{N} \times \mathbb{N} \times \ldots , \leq)$

\begin{align*}
  \mu(n) =& \mu_{|}(1,n) = \mu_{\leq}(0,g_1) \leq \mu_{\leq}(0,g_2) \ldots * \mu_{\leq}(0,g_r) \\
    =& \begin{cases} 
      1 & \text{if } n=1 \\
      (-1)^r & \text{if } n = p_1 p_2 \ldots p_r \text( square-free) \\
       & \text{if } \exists p\in \mathbb{P}: p^2|n
     \end{cases} \\
  m \not| n: &\mu(m,n) = 0 \\
  m \mid n:  &\mu(m,n) = \mu(1, \frac{n}{m}) \\
\end{align*}

\TODO{Hassediagram, [3,18], show when m|n we can divide}

\begin{align*}
  f: &\mathbb{N}^{+} \rightarrow \mathbb{R}, \\
  S_f(n) =& \sum_{d\mid n} f(d) \Rightarrow f(n) = \sum_{d\mid n} S_f(d)\mu\left(\frac{n}{d}\right)
\end{align*}

\subsection{Lattices}
\begin{definition}
Let $(P, \leq)$ be a poset, and $x,a,b \in P: a\leq x \leq b$. Furthermore say $a$ is a lower bound for $x$, and $b$ be an upper bound for $x$.

The smallest common upper bound of $x,y \in P: x \vee y$ ("$x$ join $y$").

The greatest common lower bound of $x,y \in P: x \wedge y$ ("$x$ meet $y$").

\begin{align*}
    X \subseteq P: &\bigvee_{x \in X} x \text{ join of } X, 
        & \bigwedge_{x \in X} x \text{ meet of } X 
\end{align*}
\end{definition}

\begin{definition}
A poset $(L, \leq)$ is a \dt{lattice} if $\forall x,y \in L \; \exists x \wedge y, x \vee y$.

If only $x \wedge y$ exists: meet-semilattice. 

If only $x \vee y$ exists: join-semilattice. 

\TODO{diagramm of lattice}

If $\forall X \subseteq L : \bigwedge_{x \in X} x, \bigvee_{x \in X} x$ exist, $\rightarrow$ complete lattice
\end{definition}

\textbf{Example.}
\begin{align*}
  (2^M, \subseteq) \\
  A\vee B = A \cap B, \\
  A\wedge B = A \cup B
\end{align*}

\begin{align*}
  (\mathbb{N}, \mid) \\
  x \vee y = gcd(x,y), \\
  x \vee y = lcm(x,y)
\end{align*}

\textbf{Remember.}
$(L, \vee, \wedge): \vee, \wedge$ are both associative, commutative, idempotent ($a\wedge a = a$, $a\vee a = a$). 

absorbtion laws: $a \wedge(a \vee b) = a = a \vee ( a \wedge b)$, $0 \vee a = a$, $1 \wedge a = a$. 

Furthermore we have semigroups $(L, \wedge)$ and $(L, \vee)$.

\Lemma.
If $L$ is a lattice, and $x,y,s,t \in L$. And $s$ is a common upper bound $x \leq s, y \leq s \Rightarrow x \vee y \leq s$. 
And $x \geq t, y \geq t \Rightarrow x \wedge y \geq t$. 

\Remark.
\[
  a \leq b \Leftrightarrow a \vee b = b \Leftrightarrow a \wedge b = a
\]

\Lemma.
$L$ finite meet-semilattice with 1-element. Then $L$ is a lattice.

\Proof.
Assume $x,y \in L$, Set $B = \{u \in L \mid x \leq u \wedge y \leq u\}$. 

\begin{align*}
  x \leq 1, y \leq 1
    \Rightarrow 1 \in B \Rightarrow B \neq \varnothing, |B| < \infty \\
  \Rightarrow B= \{u_1, u_2 \ldots , u_m\}: 
    u:= u_1 \wedge u_2 \wedge \ldots \wedge u_m \\
  \Rightarrow u \in B \\
  u_i \geq x, u_i \geq y \Rightarrow u_i \geq x \wedge y
\end{align*}


\Lemma.
\begin{align*}
  &x \leq s, y \leq s \\
  &x \leq t, y \leq t \\
  &\Rightarrow x \leq s \wedge t, y \leq s \wedge t
\end{align*}

\Proof.
$s \wedge t$ greatest common lower bound of $s,t$. 
$x,y$ are lower bounds of $s,t$ $\Rightarrow$ $x \leq s \wedge t, y \leq s \wedge t$. 

\textbf{Example.}
$\Pi_n$ is the set of partitions of $\{1,2,..., n\}$. 
\[
    (\Pi_n, \leq): A \leq B: \iff \text{$A$ is refinement of $B$}
\]
\TODO{Table of refinement of B with fancy coloring}

We claim $(\Pi_n, \leq)$ is a lattice. 
It is easy to show that $(\Pi_n, \leq)$ is a poset. 

\begin{align*}
  \text{1-el}: & \text{ 1-block partition} \\
  \text{0-el}: & \text{ each block of size 1} \\
  \alpha, \beta \in \Pi_n: & \;\alpha \wedge \beta: i,j \in \{1,2,\ldots,n\} 
    \text{ in the same block } \\
    & \implies \text{ they are in the same block of both $\alpha$ and $\beta$} \\
  &\Rightarrow \Pi_n \text{ is a lattice}
\end{align*}

\Theorem.
$L$ is a lattice with 0-element and 1 element, $b \in L\backslash \{1\}$
\begin{align*}
    \Rightarrow \mu(0,1) = -\sum_{x: x\wedge b = 0, x \neq 0} \mu(x,1) (*)
\end{align*}

\Proof.
$(*) \Leftrightarrow \sum_{x: x\wedge b = 0} \mu(x,1) = 0$
, since $0 \wedge b = 0$

Let $y \leq b$, and we set $N(y) = \sum_{x: x\wedge b=y} \mu(x,1)$.

\begin{align*}
  S(b) &:= \sum_{y: y \leq b} N(y) \\
    & = \sum_{y: y \leq b} \sum_{x: x\wedge b=y} \mu(x,1) \\
    & = \sum_{x,y: x \wedge b = y} \mu(x,1)
\end{align*}

\textbf{Note.}
\begin{align*}
  &\forall x \in L \; \exists!\, y \in L : x \wedge b = y \\
  &\Rightarrow \forall x \in L: \mu(x,1) \text{ occurs exactly once in $S(b)$} \\
  &\Rightarrow S(b) = \sum_{x \in L} \mu(x,1) = \sum_{x \in [0,1]} \mu(x,1) = \delta_{0,1} = 0 \\
  &\Rightarrow \sum_{y \in [0,b]} N(y) = 0 \\
  &\stackrel{\text{MIF}}{\Rightarrow} N(b) = \sum_{y \leq b} S(y) \mu(y,b) = 0\\
  &N(0) = 0, \text{ but } N(0) = \sum_{x:x \wedge b = 0} \mu(x,1) = 0 \\
\end{align*}

With the Note in mind, we can now compute for the previous example
\[
    \mu( 0_{\Pi_n}, 1_{\Pi_n}) = (-1)^{n-1} (n-1)!
\]


\chapter{Number Theory}

\input algebraic_structures.tex

\section{Divisibility and Factorisation}

\begin{definition}
$a,b \in \mathbb{Z}: a|b: \Leftrightarrow \exists c \in \mathbb{Z}: a * c = b$

More general: 
$a,b \in R$, and $R$ is a ring (e.g. polynomials $R = \mathbb{Z}[x]$)
\[
  a|b \Leftrightarrow \exists c \in R , a * c = b
\]
\end{definition}

\begin{definition}
$a,b \in \mathbb{Z}$, $d = \gcd(a,b)$
\begin{align*}
    \Leftrightarrow & \text{ 1) } d|a \wedge d|b \\
    \Leftrightarrow & \text{ 2) } t|a \wedge t|b \Rightarrow t|d \\
\end{align*}
\end{definition}

\begin{definition}
division with remainder
\begin{align*}
  a,b \in \mathbb{Z}, \; 
  b > 0 \;\exists q,r: 
    & \; a = b q  + r, \\ 
    & \; 0 \leq r < b
\end{align*}
\end{definition}



