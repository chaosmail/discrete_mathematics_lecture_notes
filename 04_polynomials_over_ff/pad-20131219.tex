%%% EtherPad for Discrete Mathematics VO
%%% http://www.informatik-forum.at/showthread.php?104454-Notes-2013WS-VO_01
%%% Past pads:
%%%     * 2013-10-17: committed by patrikf
%%%     * 2013-10-18: committed by patrikf
%%%     * 2013-10-24: committed by patrikf
%%%     * 2013-10-25: committed by patrikf
%%%     * 2013-11-08: committed by neroburner
%%%     * 2013-11-21: committed by neroburner
%%%     * 2013-11-22: committed by neroburner
%%%     * 2013-11-28: committed by neroburner
%%%     * 2013-11-29: committed by neroburner
%%%     * 2013-12-05: committed by neroburner
%%%     * 2013-12-06: committed by neroburner
%%%     * 2013-12-12: committed by neroburner
%%%     * 2013-12-13: committed by neroburner

% Discrete Mathematics Lecture Notes 2013-12-19

\textbf{Remember from last time.}
\lecturedate[\baselineskip]{2013-12-19}
There are rings, where we have unique factorizations into prime elements.
Every Euclidean ring is a factorial ring. Divisibility theory, Euclidean Algorithm.

We need to go deeper.

\subsubsection{Ideals in rings}

First some facts from proof-theory

\textbf{Recall.}
$(G,*)$ and subgroup $U \leq G: a * U = \{ a * x \mid x \in U \}$. These sets form a partition of the group.
$ a \neq a' \begin{cases} a*U = a'*U $ or$\\ (a*U) \cap (a'*U) = \varnothing \end{cases}$

\begin{tabular}{ll}
  left cosets  & $a*U$, $a \in G$\\
  right cosets & $U*a$
\end{tabular}


If $U \leq G$ such that $\forall a \in G: a*U = U*a$ then $U$ is called \dt{normal subgroup} $(U \NormSubgroup G)$

$U \NormSubgroup G :$ $\underbrace{(a*U)}_{=(a' *U)} * \underbrace{(b*U)}_{=(b' *U)} = \underbrace{(a*b) * U}_{=(a'*b')*U}$

$(G / U, *)$ is called a \dt{quotient group}

$G/U = \{ a*U \mid a \in G\}$ \quad (read: ``G modulo U'')

\begin{definition}
  $R,S$ ring, $\varphi : R \rightarrow S$

  (ring) homomorphism if
  \begin{align*}
    \varphi(a+b) &= \varphi(a) + \varphi(b)\\
    \varphi(a \cdot b) &= \varphi(a) \cdot \varphi(b)\\
  \end{align*}

  kernel of $\varphi$: $\Kernel \varphi = \{ x \in R \mid \varphi(x) = 0\}$
\end{definition}

\Theorem.
\begin{align*}
(\Kernel \varphi, +) \NormSubgroup (R, +) \text{ and } a \cdot \Kernel \varphi &\subseteq \Kernel \varphi\\
(\Kernel \varphi) \cdot a &\subseteq \Kernel \varphi
\end{align*}
\begin{align*}
  &x \in \Kernel \varphi \Rightarrow \varphi(ax) = \varphi(a) \underbrace{\varphi(x)}_{=0} = 0 \\
  & \Rightarrow a x \in \Kernel \varphi \text{, } x a \in \Kernel \varphi
\end{align*}

\begin{definition}
  $R$ ring, $I \subseteq R$ \dt{ideal} if
  \begin{enumerate}
    \item $(I, +)$ is (normal) subgroup of (R,+), which is an abelian group
    \item $a \cdot I \subseteq I$, $I \cdot a \subseteq I$
  \end{enumerate}
\end{definition}

The ``normal'' is in parentheses because every subgroup of an abelian group is a normal subgroup.

\Remark.
$\varphi: R \rightarrow S$ homomorphism $\implies$ $\Kernel \varphi$ is an ideal of $R$

\Theorem.
$R$ ring, $ I \subseteq R$ ideal. Then define $+$ and $\cdot$ on $\underbrace{R / I}_{=(R,+)/(I,+)}$ as follows:
\begin{align*}
  (a+I) + (b+I) &\coloneqq (a+b)+I, \\
  (a+I) \cdot (b+I) &\coloneqq (ab) + I
\end{align*}

Then $(R / I, +, \cdot)$ is a ring, the \dt{quotient ring} $R$ modulo $I$.

\textbf{Example.}
$R = \mathbb{Z}, I = n*\mathbb{Z} = \{n \cdot z \mid z \in \mathbb{Z}\}$ is an ideal
\begin{align*}
  U \subseteq G \Leftrightarrow & ~1)~ U \neq \varnothing\\
                                & ~2)~ a,b \in U \Rightarrow a*b^{-1} \in U\\
  x,y \in n\mathbb{Z} \implies  & x = k\cdot n, y = l\cdot n\\
                      \implies  & x -y = (k-l) \cdot n \in n\mathbb{Z}\\
                      \implies  & n\mathbb{Z} \NormSubgroup \mathbb{Z}:
                                  a \in \mathbb{Z} : a+n\mathbb{Z} =
                                  \{ a+nz \mid z\in \mathbb{Z} \}
\end{align*}
\begin{align*}
  a+n\mathbb{Z} = \bar{a} \text{ in } \mathbb{Z}_n \\
  a \cdot n \mathbb{Z} \subseteq n \mathbb{Z}
\end{align*}
\begin{align*}
  \underbrace{\mathbb{Z} / n \mathbb{Z}}_{=\mathbb{Z}_n}:
    & \bar{a} + \bar{b} = \overline{a+b}, \bar{a} \bar{b} = \overline{ab} \\
    & a+b \equiv a+b (n), a\cdot b \equiv a\cdot b (n)
\end{align*}

\Remark.
$R$ ring $\implies \{0\}$ and $R$ are the trivial ideals

\begin{tabular}{l|l}
+ & 0 \\
\hline
0 & 0
\end{tabular}
, $a \cdot \{0\} \subseteq \{0\}$

\begin{definition}
  $R$ ring, $I \subseteq R$ ideal

  We then define an equivalence relation $\sim$ on $R$ (i.e. $\sim \subseteq R \times R$)

  $\sim$ is compatible with $+$ and $\cdot$ if
  \begin{align*}
    a \sim b, c \sim d \implies & a +c \sim b+d \\
                                & a \cdot c \sim b \cdot d
  \end{align*}

  $\sim$ equivalence relation which is compatible with $+$ and $\cdot$
  is called \dt{congruence relation}, e.g $\equiv \mod n$

  In particular: $a\sim b :\Leftrightarrow a + I = b + I \Rightarrow \sim$ is a congruence relation
\end{definition}

\Theorem.
$R$ ring with $1$, $I \subseteq R$ ideal, $ \epsilon \in R^{*}, \epsilon \in I$

Then $R = I$

\Proof.
$\epsilon \in I \cap R^{*} \implies \exists \epsilon^{-1}$ \\
$\forall r \in R: r \cdot I \subseteq I$, in particular $\epsilon^{-1} \cdot \underbrace{I}_{\epsilon \in I} \subseteq I \implies 1 \in I$

$r \cdot I \subseteq I \implies r \cdot 1 = r \in I \implies R \subseteq I \implies R = I$

By definition we have $I \subseteq R$. In the proof we showed $R \subseteq I$. This implies $R = I$.

\Corollary.
A field $K$ has only the trivial ideals $\{0\}$ and $K$

\Remark.
$R$ ring, $(I_j)_{j\in J}$ family of ideals $\implies \bigcap_{j \in J} I_j$ is an ideal as well.

\begin{definition}
  $M \subseteq R$, $R$ ring,
  \[
    (M)\coloneqq \bigcap_{I \subseteq R, I \text{ ideal}, M \subseteq I} I
  \]
  is the \dt{ideal generated by M}. Which is the smallest ideal with contains $M$ (with respect to $\subseteq$).
\end{definition}

\begin{definition}
  An ideal generated by one element $a$ $\left( =(a) \right)$, is called a \dt{principal ideal}.
\end{definition}

\Theorem.
$R$ Euclidean ring. Then every ideal is a principal ideal.
i.e. $R$ is a \dt{principal ideal domain}.

\Remark.
Euclidean ring $\subseteq$ principal ideal domain $\subseteq$ factorial ring $\subseteq$ integral domain

\Example.
1) $R=\mathbb{Z}: (n) = n\cdot \mathbb{Z}$

$M = \{m_1, m_2, \ldots, m_k\} \subseteq \mathbb{Z}, (M)$\\
\begin{align*}
  m_1, m_2 &\implies a \cdot m_1 + b\cdot m_2 \in (M) \\
           &\implies \gcd(m_1,m_2) \cdot \mathbb{Z} \subseteq (M) \\
           &\implies M \subseteq gcd(m_1,m_2) \cdot \mathbb{Z} \\
           &\implies (M) = \gcd(m_1,m_2) \cdot \mathbb{Z}
\end{align*}

\Example.
2) $x\in R^{*} \implies (x) = R$

\Example.
3) $\mathbb{Q}$ is a subring (even a subfield) of $\mathbb{R}$, but no ideal

\subsection{Fields}

\textbf{Recall.} Properties of a Field:
\begin{itemize}
  \item $\underbrace{(K,+)}_{\text{neutral el. }= 0}$ is an abelian group

  \item  $\underbrace{(K\backslash \{0\}, \cdot)}_{\text{neutral el. }= 1}$ is an abelian group

  \item $\forall a,b,c \in K: a(b+c) = ab+ac, (a+b)c = ac+bc$

  \item $0\cdot x = (0+0) \cdot x = 0x + 0x \quad | -0x$\\
        $0 = 0x$\\
        $x0 = 0$ similarly
\end{itemize}

$(K_i)_{i\in I}$ family of subfields $\implies$ $\bigcap_{i\in I} K_i$ is a subfield

\begin{definition}
  $\displaystyle{\bigcap_{K' \text{ subfield of $K$}} K'}$ is \dt{prime field} of $K$, denoted by $P(K)$
\end{definition}

$\{0\}$ is not a field, because in every field we have that $0 \neq 1$ ($\implies$ every field has at least two elements)

\begin{definition}
  $\ord_{(K,+)} (1)$ is the \dt{characteristic of K} ($\Char K$)
\end{definition}

$1,1+1, 1+1+1, \ldots, \underbrace{1+1+ \ldots +1}_{\Char K} = 0$ if this is finite.

$\Char K = 0$ if $\ord_{(K,+)} (1) = \infty$

\Example.
\begin{align*}
  (\mathbb{R}, +): &\Char \mathbb{R} = 0\\
  &\Char \mathbb{Q} = 0\\
  &\Char \mathbb{C} = 0\\
  (\mathbb{Z}_2, +, \cdot): &\Char \mathbb{Z}_2 = 2\\
  &\Char \mathbb{Z}_p = p \text{ if } p \in \mathbb{P}
\end{align*}

\textbf{Properties of $P(K)$}
Case 1:
$\Char K=0$:
$\forall K'$ subfield of $K$: $0,1 \in K' \implies 0,1 \in P(K)$

\begin{align*}
&1, 1+1, 1+1+1, \ldots = k-1, k \in \mathbb{N}\\
&-1, (-1)+(-1), \ldots = k(-1) = -(k\cdot 1) = (-k) \cdot 1\\
&(k\cdot 1)^{-1} \in P(K)\\
&k\cdot 1, (-k)\cdot 1,(k\cdot 1)^{-1}, (k\cdot 1)\cdot (l\cdot 1)^{-1} \in P(K), \quad k \in \mathbb{Z}, l \in \mathbb{N} \backslash \{0\}\\
&\{ k \in \mathbb{Z}, l \in \mathbb{Z}, l>0 \} \cong \mathbb{Q}\\
&\implies P(K) \cong \mathbb{Q} \implies |K|=\infty
\end{align*}

Case 2: $\Char K \neq 0$

\Lemma.
\begin{align*}
  p = \ord_{(K,+)} (1) < ∞ \implies & \text{1) } ∀ a ∈ K ∖ \{0\}:
                                      \ord_{(K,+)} (a) = p\\
                                    & \text{2) } p \in \mathbb{P}
\end{align*}

\Proof.
\begin{enumerate}[1)]

  \item $p\cdot 1 = \underbrace{1+1+ \ldots +1}_{p \text{ times}} = 0$
  \begin{align*}
    \implies & p \cdot a = \underbrace{a+a+ \ldots +a}_{p\text{ times}} = a \cdot \underbrace{(1+1+ \ldots +1)}_{=0} = a\cdot 0 = 0 \\
    \implies & \ord(a) \leq p
  \end{align*}

  Assume $\ord(a) = m$
  \begin{align*}
    \implies
    & \overbrace{(m\cdot a)}^{=0}\cdot a^{-1} =
      \underbrace{\overbrace{a\cdot a^{-1}}^{=1} + \overbrace{a\cdot a^{-1}}^{=1} + \ldots + \overbrace{a\cdot a^{-1}}^{=1}}_{m\text{ times}} = m\cdot 1 \\
    \implies & m \geq p \implies m = p
  \end{align*}

  \item
  \begin{align*}
  p=a\cdot b & \implies 0=p \cdot 1 = \underbrace{\underbrace{1 + 1 + \ldots + 1}_{a\text{ times}} + a\cdot 1 + a\cdot 1 + \ldots + a\cdot 1}_{b\text{ times}}\\
  &\implies b \cdot (a \cdot 1) = a\cdot 1 + a\cdot 1 + \ldots + a\cdot 1\\
  &\implies \ord(a\cdot 1)=b < p \quad \text{Contradiction!}\\
  \Char K=p &\implies P(K) \cong \mathbb{Z}_p
  \end{align*}

\end{enumerate}
