% Discrete Mathematics Lecture Notes 17.10.2013

% (Last week)
Flow network: G = (V,E,w,s,t)
V ... vertex set
E ... edge set
w ... weight function
s ... sourcee
t ... sink

Flow function $\phi$
size (valuation of $\phi$): $v(\phi)$

% new content

\begin{definition}
Cut of G is a partition of $V=S \cup T$ ($S\cap T=0$) such that
$s \in S,t \in T$
\end{definition}


Capacity $c(S,T)$ of a cut:
\[ c(S,T) = \sum_{x,y x\in S, y\in T} w(xy) \]

Cut $(S,T)$ is minimal if all cuts $(S',T')$ satisfy:
\[ c(S', T') \geq c(S,T) \]


Lemma: Let G flow network, $\phi$ flow on $G$, $(S,T)$ cut of G. Then
\[
v(\phi) = \sum_{xy: x\in S, y\in T} \phi(xy)
        - \sum_{xy: x\in T, y\in S} \phi(xy)
\leq c(S,T)
\]
(flow forward minus flow backwards)
in particular: $\max_{\phi flow} v(\phi)\leq min_{(S,T) cut} c(S,T)$

Proof.
\[
    v(\phi) = \sum{v\in S} (
        \sum_{x\in \Gamma^+(v)} \phi(vx)
      - \sum_{y\in \Gamma^-(v)} \phi(yv)
    )
\]
For each vertex, count outgoing edge with positive and incoming edge with negative sign.

3 cases:
* Flow with both endpoints in S; outgoing and ingoing cancel each other
    \TODO: add picture of edge
* Flow with starting point in S and end point in T, counted positive
* Flow with starting point in T and end point in S, counted negative

Thus, we can rewrite $v(\phi)$:
\[
    v(\phi) = \sum_{x,y: x\in S, y\in T} \phi(xy)
            - \sum_{x,y: x\in T, y\in S} \phi(xy)
\]
Second sum is nonnegative, $0 \leq \phi (xy) \leq c(S,T)$
First sum: <= w(x,y), $\phi (xy) \leq w(xy)$

Thus
\[
    v(\phi) \leq c(S,T).
\]



\begin{definition}
A path $P$ $s\rightarrow t$ (edge, or walk without repeated edges, irrespective of edge direction) is called \dt{augmenting path} with respect to a flow $\phi$ if $\phi(e) < w(e)$ on every forward edge of $P$ and $\phi(e) > 0$ on every backward edge.
\end{definition}

\textbf{Theorem.}
Let $G$ flow network, $\phi$ flow on $G$. Then $\phi$ maximal with respect to the size ($v(\phi)$ max) $\Leftrightarrow \nexists$ augmenting path (w.r.t. $\phi$).

\textbf{Proof:}
"$\Rightarrow$"
Suppose there is an augmenting path $P$. Then we choose
\begin{align*}
    \delta' &= min_{e \in P, forward} w(e) - \phi(e) > 0 \\
    \delta'' &= min_{e\in P, backward} \phi(e) > 0 \\
    \delta &= min(\delta', \delta'') > 0
\end{align*}

Define
\[
    \tilde\phi := \begin{cases}
        \phi(e) + \delta &\text{if $e$ forward edge of $P$} \\
        \phi(e) - \delta &\text{if $e$ backward edge of $P$} \\
        \phi(e) &\text{otherwise}.
    \end{cases}
\]

$\tilde\phi$ is a flow on $G$.

Because we take the minimum of all the flows $\delta$, $\tilde\phi$ does not become negative on backward edges.

Because we increase the flow on the augmenting path by $\delta$,
\[
    v(\tilde\phi) = v(\phi) + \delta > v(\phi)
\]

"$\Leftarrow$" Assume there is no augmenting path.
\[
    S = \{v\in V | \exists augmenting path s\rightarrow v\}
\]
There is no augmenting path $s\rightarrow t$, thus $t \notin S$.
Thus $(S, T = V\setminus S)$ is a cut.
Each edge $xy$ with $x\in S, y\in T$ must be saturated: otherwise it would be possible to find an augmenting path.

\TODO add picture of S patch x edgy y in T

Each edge $xy$ with $x\in T, y\in S$ must be void ($\phi(xy) = 0$). Thus
\[
    v(\phi) = c(S,T) - 0.
\]

Therefore $\phi$ must be maximal.

\textbf{Theorem.} $G$ is a flow network such that
\[
    \forall e\in E': w(e)\in \mathbf{N} \implies \exists \text{maximal flow}.
\]

\textbf{Proof.} $ \phi_{0}(e) \equiv O$, if $\phi_{0}$ not max $\implies \exists$ augmenting path with $\delta \in \mathbf{N^{+}}$

Construct $\phi_1$ according to previous proof ($\tilde\phi$). We know that $v(\phi_1) ≥ 1$. Now we iterate and get $\phi_2, v(\phi_2) ≥ 2$,~\ldots After finitely many steps, we reach a flow $\phi_{\text{max}}$ where we cannot find an augmenting path anymore.

\textbf{Corollary.} If $G$ is a flow network, weight function $w: E\mapsto \mathbf{Q}$, this implies $\exists$ maximal flow.

\textbf{Remark.}
If weights are real, it can be shown that a maximal flow exists (but you need a different proof strategy).

\textbf{Theorem (Max-flow/min-cut theorem).}
Let $G$ be a flow network $\implies \exists$ max. flow and $v(\phi_{\text{max}})= \min_{(S,T) cut} c(S,T)$

% slides: Algorithm of Ford-Fulkerson


\section{Special Graph Classes}

\subsection{Eulerian Graphs}

\begin{definition}
A \dt{Eulerian trail}, which can be open or closed (\dt{Eulerian tour}), is a trail where no edge is repeated and every edge is used. Closed: start = end
\end{definition}

\begin{definition}
A graph is called \dt{Eulerian} if there is an Eulerian tour.
\end{definition}

\textbf{Theorem.}
Let $G=(V,E)$ be an undirected, connected multigraph. Then $G$ is Eulerian $\iff \forall x\in V: d(x)\text{ even}$.

There is an \emph{open} Eulerian trail $\iff$ exactly two vertices have odd degree.

\textbf{Proof.}
% "<==" direction
Proof by induction on $\alpha_1(G) =: m$.

$m = 0$: Trivial, only one vertex.

$m ≥ 1$: construct a tour $W$. You can always find a tour because degrees are even, thus every time you enter a vertice you can certainly find an edge where you can leave it.

Two cases:
a) $W$ is an Eulerian tour
b) $W$ is not an Eulerian tour

Then remove $W$, resulting in a graph $G'$ with connected components $G'_1$,~\ldots Then
\[
    \forall x\in V(G'_{i}): d_{G'}(x)\text{ is still even}
\]

Since all $G'_{i}$ are smaller than $G$, we can apply the induction hypothesis. This means $G'_{i}$ must be Eulerian (with Eulerian tours $W_{i}$.

$\forall i: W_i$ and $W$ have common vertex. Thus you can construct an Eulerian tour for the whole graph.

% Slides: Eulerian tour (started at 09:57... :)

% "==>" direction

Walk on an Eulerian tour. 











