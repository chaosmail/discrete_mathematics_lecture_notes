%%% EtherPad for Discrete Mathematics VO
%%% http://www.informatik-forum.at/showthread.php?104454-Notes-2013WS-VO_01
%%% Past pads:
%%%     * 2013-10-17: committed by patrikf
%%%     * 2013-10-18: committed by patrikf
%%%     * 2013-10-24: committed by patrikf
%%%     * 2013-10-25: committed by patrikf
%%%     * 2013-11-08: committed by neroburner
%%%     * 2013-11-21: committed by neroburner
%%%     * 2013-11-22: committed by neroburner
%%%     * 2013-11-28: committed by neroburner
%%%     * 2013-11-29: committed by neroburner
%%%     * 2013-12-05: committed by neroburner
%%%     * 2013-12-06: committed by neroburner
%%%     * 2013-12-12: committed by neroburner

% Discrete Mathematics Lecture Notes 2013-12-13

\section{Polynomial over Finite Fields}
\lecturedate[\baselineskip]{2013-12-13}

\subsection{Rings}
\begin{definition}
Is called a \dt{Ring} $(R,+,\cdot)$ if
\begin{tabular}{ll}
  $(R,+)$ & abelian group (neutral element: 0)\\
  $(R,\cdot)$ & semigroup \\
  $\forall ab,c \in R:$ & $a(b+c) = ab+ac$\\
                        & $a+b)c = ac+bc$\\
\end{tabular}
\end{definition}

\begin{definition}
An \dt{integral domain} in a \dt{commutative ring} with $1$. 
and no zero divisors: $a\cdot b = 0 \implies a=0 \lor b = 0$
\end{definition}

\begin{definition}
An \dt{Euklidean ring} is an integral domain with an Euklidean function
\[
  n: R \backslash \{0\} \rightarrow \mathbb{N} \text{ s.t. }
  \forall a,b \in R: \exists q,r \in R: a 0 bq + r, n(r) < n(b) \lor r=0
  \text{ and } n(a) \leq n(ab)
\]
\end{definition}

Let us assume we have an integral domain $(R,+, \cdot)$. 
$t|a: \Leftrightarrow \exists c: a=tc =ct$
\begin{tabular}{ll}
$d=\gcd(a,b) if$ & 1) $d|a \land d|b$ \\
  & $t|a \land t|b \implies t|d$
\end{tabular}

\textbf{Remember}
in $\mathbb{Z}$: $d = gcd(a,b) \implies -d = \gcd(a,b)$

\begin{definition}
$R$ integral domain, 
$a,b \in R$ are called associated, $a \tilde b$,
\[
  \Leftrightarrow: \exists \text{ unit } r \in R^{*}: a = rb
\]
\end{definition}

Recall the set (group) of units:
\[
  R^{*} = \{ x \in R \mid \exists x^{-1} : xx^{-1} = 1\}
\]
\begin{tabular}{ll}
  $(R^{*}, \cdot)$ \ldots 
    & group: $1^{-1} = 1 \implies 1 \in R^{*}$ \\
    & $x\in R^{*} \implies x^{-1} \in R^{*}$
\end{tabular}

\Theorem.
$R$ is an Euklidiean ring, $a,b\in R, a|b \implies n(a) \leq n(b)$

\Proof.
\begin{align*}
  \exists c : b = ac, n(a) \leq n(ac) = n(b)
\end{align*}

\textbf{Corollary.}
$d, d'$ are $\gcd$'s of $a$ and $b$ $\implies n(d) = n(d')$

\Proof.
$d|d' \land d' | d$


\Remark.
1) $x = a\cdot b, a,b\not\in R^{*} \cup \{0\} \implies n(a) < n(x), n(b) < n(x)$

\Remark.
2) In general integral domains: 
\[
  d= \gcd(a,b), d' = \gcd(a,b) \implies d \tilde d'
\]
\[
  d|d' \land d'|d \implies d = c_1\cdot d', d' = c_2 \cdot d 
  \implies d = c_1c_2 *d \implies c_1 c_2 = 1
  \implies c_1,c_2 \in R^{*}
\]

\subsection{Generalization of prime numbers}
\begin{definition}
  Integral domain $R$, with $a\in R \backslash (\{0\}\cup R^{*})$, i.e. $\nexists a^{-1}$

$a$ \dt{irreducible} $:\Leftrightarrow a = bc$ implies that either $b\in R^{*}$ or $c\in R^{*}$

Sideremark: 
$\mathbb{Z}: \mathbb{Z}^{*} = \{-1, 1\}$ \\
$p = p \cdot 1 = (-p) \cdot (-1)$

$a$ \dt{prime element} $:\Leftrightarrow a | bc$ implies that $a|b$ or $a|c$
\end{definition}

\textbf{Example.}
$R = \mathbb{Z} \implies x \in R$ irreducible $\iff x \in \mathbb{P}$ or $-xin \mathbb{P} \iff $ $x$ prime element

\Theorem.
\begin{enumerate}
  \item Every prime element is irreducible
  \item In Euklidean rings the convers is true as well
\end{enumerate}

\Proof.
1) a is a prime element and $a = bc$ (which implies $a |b$ or $a|c$)\\
If $a|b$ then $a|b \land b|a \implies a = bc, b = a\bar{c} = b c\bar{c}$\\
$b(1-c\bar{c}) = 0 \implies c\bar{c} = 1 \implies c, \bar{c} \in R^{*} 
\implies a \sim b$

2) a irreducible, $a|bc$ 

Suppose $a \not|: d = \gcd(a,b): a = d\cdot c_1,\; b = d\cdot c_2$.
Either $d$ or $c_1$ has to be a unit. 
If $c1 \in R^{*} = \stackrel{\implies}{\exists c_1^{-1}} d = a\cdot c_1^{-1} \implies b = a \cdot c_1^{-1} \cdot c_2 \implies a |b$ Contradiction! \\
$\implies d \in R^{*}$: 
Without loss of generality: $d = 1$\\
we are in an Euklidean Ring $R$ $\implies$ $\exists x,y \in R: 1 = ax + by$ \\
$\implies c = acx + bcy = acx + ay = a (cx+y) \implies a|c$

In the same way we can proof $a\not|c \implies a|b$.

\textbf{Example.}
$R = \mathbb{Z} [i \sqrt{5}] = \{a+bi \sqrt{5} \mid a,b \in \mathbb{Z} \} \subseteq \mathbb{C}, i = \sqrt{-1}$

$R(,+,\cdot) \underbrace{\leq}_{\text{subring}} (\mathbb{C}, +,\cdot)$
\begin{align*}
  (a+bi \sqrt{5}) (c+di\sqrt{5}) = 
    \underbrace{ac - 5bd}_{\in \mathbb{Z}} + \underbrace{(ad+bc)}_{\in \mathbb{Z}} i\sqrt{5} \in \mathbb{Z}[i\sqrt{5}]
\end{align*}

\begin{align*}
  1+0 * i\sqrt{5} \in R\\
  6 = 2*3 = (1+i\sqrt{5})(1-i\sqrt{5})\\
\end{align*}
\begin{align*}
  2|6, 2 \not| 1+i\sqrt{5}: & 1+i\sqrt{5} = 2c = 2(a+bi\sqrt{5}) \\
    & \implies 2a = 1, 2b = 1 \\
    & \implies a \not\in \mathbb{Z}
\end{align*}

$2 \not| 1 - i\sqrt{5}$ but $2 | (1+i\sqrt{5})(1-i\sqrt{5})$ \\
$\implies 2$ is not a prime element of $R$

\begin{align*}
  2 = \underbrace{(a+bi \sqrt{5})}_{r} \underbrace{(c+di\sqrt{5})}_{s}\\
  \implies \frac{2}{a+ib\sqrt{5}} \frac{a-i\sqrt{5}b}{a-i\sqrt{5}b} \\
  = c+di\sqrt{5} 
= \underbrace{frac{2a}{a^2 + 5b^2}}_{\in \mathbb{Z}} - \underbrace{i\sqrt{5} \frac{2b}{a^2 + 5b^2}}_{\in \mathbb{Z}}
\end{align*}

But $a^2 + 5b^2 > 4b^2 > 2|b|$ except if $ b=0$ $\implies d \not\in \mathbb{Z}$ except if $b=0$ \\
If $b=0: c = \frac{2a}{a^2} = \frac{2}{a} \implies a \in \{\pm 1, \pm 2\}$
$\implies r = \pm 1 \implies r \in R^{*}$ or $ r = \pm 2 \implies s in R^{*} \implies 2 $ irreducible

$\implies \mathbb{Z}[i\sqrt{5}]$ is not an Euklidian ring

\textbf{Example.}
$K$ is a field \\
\begin{tabular}{ll}
  $(K,+)$ &abelian group \\
  $(K\backslash \{0\}, \cdot)$ & abelian group \\
  $a(b+c) = ab + ac$
\end{tabular}
The previous properties imply $\forall x \in K: 0\cdot x = 0$

Recall: 
$n(p(x)) = \deg(p(x))$ and 
$\deg(p(x) \leq \deg(p(x) q(x))$. $\implies$ The prime elements are the irreducible elements: irreducible polynomials: $a(x) = b(x)c(x)$

\begin{align*}
  &r(x) \in K[x]^{*}, 1 = 1*x^0 + 0 x^1 + \ldots \\
  &r(x) \in K[x]^{*} \Leftrightarrow r(x) \neq 0, \deg(r(x)) = 0\\
\end{align*}

in $\mathbb{C}[x]: 
\deg(p(x)) = n \implies \exists n$ not necessarily different zeros $a_1, a_2, \ldots a_n$ and $p(x) = (x-a_1)(x-a_2) \ldots (x-a_n)$

$\implies p(x)$ is irreducible $\iff$ $p(x) = ax+b$ linear polynomial

in $\mathbb{R}[x]:$
$x^4 + 1 = (x^2 + \sqrt{2} +1)(x^2 - \sqrt{2} + 1)$

\begin{tabular}{ll}
  irr.: 
    & $ax+b$ and \\
    & $ax^2 + bx + c$ without zeros\\
\end{tabular}

if you find a complex zero polynomial \\
$p(a) = 0 \implies p(\bar{a}) = 0$ \\
$(x-a)(x-\bar{a}) = x^2 - \underbrace{(a + \bar{a})}_{Re(a)} x + \underbrace{a\bar{a}}_{|a|^2}$

\begin{definition}
integral domain $R$ such that $\forall a \in R \backslash (\{0\} \cup R^{*})$ 
there exists a unique representation $a = \epsilon p_1 p_2 \ldots p_k$ 
where $\epsilon \in R^{*}$ and $p_1, \ldots, p_k$ are prime elements.

unique: $\epsilon p_1 \ldots p_k = \eta q_1 \ldots q_k \implies k=l$ and there is a permutation $\pi$ such that $p_i \sim q_{\pi(i)} \forall i = 1,\ldots, k$

$15 = 1*3*5 = (-1) * (-5) * 3$, $ 3 \sim 3, 5 \sim (-5): 5=\underbrace{(-1)}_{\mathbb{Z}^{*}} (-5)$

Then $R$ is called a \dt{factorial ring} (ZPE-Ring). 
\end{definition}

\Theorem.
Every Euklidean ring is a factial ring.

\Proof.
We have to show existence and uniqueness. 

Existence:\\
Case1: 
$a$ irreducible $\Leftrightarrow$ $a$ prime elment $\implies$ $a=1\cdot a$. This is a representation as desired. 

Case2: $a= bc, b,c \not\in R^{*}$ $\implies$ $n(b) < n(a), n(c) < n(a)$

Suppose that a does not have a reprsentation of the form (*) and $n(a)$ is minimal. 
This implies that $b$ must have a prime repesentation $b = \epsilon_1 p_1 \ldots p_k$ $c = \epsilon_2 q_1 \ldots q_l$ 
$\implies a = \underbrace{\epsilon_1 \epsilon_2}_{\in R^{*}} p_1 \ldots pk q_1 \ldots q_l$. Contradiction!

Uniqueness:\\
$a = \epsilon p_1 \ldots p_k = \eta q_1 \ldots q_l, k\geq 2$\\
$\implies p_1 | \eta q_1 \ldots q_l \implies_{p_1 \not\in R^{*}} p_1 | q_1 \ldots q_l$ 
$\implies \exists i: p_1 |q_i$ W.l.o.g: $i = 1: p_1|q_1 \implies p_1 \sim q_1$ $p_1 = \epsilon_1 q_1$ with $\epsilon \in R^{*}$

$\epsilon p_2 \ldots p_k = \underbrace{\eta \epsilon_1}_{\in R^{*}} q_2 \ldots q_l$ W.l.o.g.: $p_1|q_w$

\TODO{finish proof properly}


