\documentclass[10pt,a4paper,titlepage,oneside,final]{book}
\usepackage[english]{babel}
\usepackage[utf8]{inputenc}
\usepackage[T1]{fontenc}
\usepackage{lmodern}
\usepackage[a4paper,textwidth=345pt,textheight=598pt,hmarginratio=1:1]{geometry}
\usepackage[final]{graphicx}
\usepackage{subfigure}
\usepackage[intlimits]{amsmath}
\usepackage{amssymb}
\usepackage{booktabs}
\usepackage{microtype}
\usepackage{float}
\usepackage[pdfborder={0 0 0}]{hyperref}
\usepackage{cleveref}
%\usepackage{siunitx}
\usepackage{tabularx}
\usepackage{caption}
\usepackage{ragged2e}
\usepackage{soul}
\usepackage{theorem}
%\usepackage{pgfplots}
%\usepackage{nameref}
\usepackage{tikz}
%\usetikzlibrary{shapes,calc,automata}
%\usepackage{rotating}
%\usepackage{listings}
%\usepackage{nicefrac}
\usepackage{makeidx}
\usepackage{paralist}

\usepackage{placeins}

\makeatletter

\captionsetup{format=hang,justification=justified,labelfont=bf,labelsep=colon,font=small}
\long\def\clap#1{\hbox to 0pt{\hss#1\hss}}

\author{Patrik Fimml, NeroBurner}
\title{Lecture Notes 2013W\\ 104.271 Discrete Mathematics VO\\ (Gittenberger)}

\makeindex
\gdef\th@custom{%
  \th@plain%
  \def\@begintheorem##1##2{%
    \item[\hskip\labelsep \theorem@headerfont ##1\ ##2.]}}%
\theoremstyle{custom}
\theorembodyfont{\normalfont}
\newtheorem{definition}{Definition}

\def\thechapter{\Roman{chapter}}

\parindent=0pt
\parskip=\medskipamount

\pltopsep=-5pt
\plpartopsep=0pt

\def\TODO{\fbox{TODO}}


\newcommand{\tikzmark}[2]{\tikz[overlay,remember picture,baseline] \node [anchor=base] (#1) {$#2$};}

\newcommand{\DrawLineHorizontal}[3][]{%
	\begin{tikzpicture}[overlay,remember picture]
		\draw[#1] (#2.west) -- (#3.east);
	\end{tikzpicture}
}
\newcommand{\DrawLine}[3][]{%
	\begin{tikzpicture}[overlay,remember picture]
		\draw[#1] (#2.north) -- (#3.south);
	\end{tikzpicture}
}

\begin{document}
% "defined term"
\def\dt#1{\textbf{#1}}

\def\remark{\noindent \textbf{Remark.}}



\maketitle
\tableofcontents
%\listoffigures
asdf \index{asdf}
% include the chapters here:
\section{Graph Theory}
\subsection{Basics}

\begin{tabular}{l l}
Graph:& $G = (V,E)$ \\

directed:& $ e \in E,$ \\
	& $ e = (v,w) \neq e=(w,v)$ \\

undirected:& $ e = {v,w} = {w,v}, e = vw$\\

loop \\

multiple Edges \\

simple Graph: & if there are no loops, no multiple edges \\


\end{tabular}

graph corresponds to a relation on V $(\leq V \times V)$

unidirectional graph $\equiv$ symmetrix relation

\subsubsection*{Notations:}
$V = V(G), E = E(G), \alpha_0 = |V|, \alpha_1 = |E|$ 

\subsubsection*{Def:}

\begin{tabular}{l l}
$v \in V$ : \\
$d(v)$		& \# edges which are incident to v \\

$d^+(v)$	& \# edges of the form $(v,w)$, out degree \\
$d^-(v)$	& \# edges of the form $(w,v)$, in degree \\

$\Gamma(v)$ & set of neighbours \\
$\Gamma^+(v)$ & set of successors \\
$\Gamma^-(v)$ & set of predecessors \\

\end{tabular}


\subsubsection*{Hand Shaking Lemma (simple graphs)}
$\sum_{v \in V} d(v) = 2 |E|$\\
$\sum_{v \in V} d^+(v) = d^-(v) = |E|$

\subsubsection*{Example: n-Hypercube:}
$G = ({0,1}^n, E)$

$vw \in E \Leftrightarrow \sum_{i=1}^{n} |v_i - w_i | = 1$ (only one switch in coordinates)

$v = v_1, v_2, \ldots , v_n$\\
$w = w_1, w_2, \ldots , w_n$

compute $a_0$ and $a_1$

$a_0 = 2^n$ \\
$a_1 = \frac{1}{2} \sum_{v \in V} d(v) = 2^{n-1} * n$

$d(v) = n \Rightarrow $ regular Graph

\subsubsection*{Def:}
if $e = vw \in E$: vw are adjacent, e and v (or w) are incident

\begin{tabular}{l l}
adjacency matrix & $A = (a_{ij})_{i,j = 1, \ldots , n}$ \\
	& $ V = \{v_1, v_2, \ldots , v_n \} $ \\
	& $ a_{ij} = \left \{ \begin{array}{l l} 1 &, v_i \sim v_j  \\ 0 & , v_i \not\sim v_j \end{array} \right. $ \\

\end{tabular}

\paragraph*{Rem:}
G unidirected $\Rightarrow$ A symmetric

\vspace{12pt}

\begin{tabular}{l l}
$A^k = (a_{ij}^{[k]})_{i,j = 1,\ldots , n}$ &

$A^k =A * A^{k-1}$ \\

$a_{ij}^{[k]} = \sum_{l=1}^n a_{il} * a_{lj}^{[k-1]} $ \\

per Definition $A^0 := I$ \\

\end{tabular}

number of connetctions between $v_i$ and $v_j$ in k steps

\subsubsection*{Def:}
\begin{tabular}{l l}
Walk \\

Trail	& no edge repeated \\

closed trail	& = circuit \\

$G=(V,E)$ \\
Subgraph $ H = (V',E')$	& s.t. $V' \subseteq V$, $E' \subseteq E$ \\
$H \subseteq G$ \\

\end{tabular}

\vspace{12pt}

\begin{tabular}{l l}
$\sim$ : adjacent relation \\

v connected to w ($vRw$) \\

$C = \sum_{k=0}^L A ^k$ 	& $ L = min(|E|, |V| -1)$ \\
$C=(c_{ij}) $ \\
$C = $\# walks of length $\leq$ L, $v_i \leadsto v_j$ \\

$M = sign(C)$ (connectivity matrix\\

$\forall v \in V: vRv$ \\
$\forall v,w \in V: vRw = wRv$ & R \ldots equivalence relation\\
$\forall v,w,u \in V: vRw \cap wRu \Rightarrow vRu$ & $ \Rightarrow$ R induces Partition of V \\

\end{tabular}


\subsubsection*{Def:}
G is connected if $\forall v,w: vRw$

\subsubsection*{Def:}
$ H \leq G$  connected component of G if H connected, H maximal and connected
$V = V_1 \cap V_2 \cap \ldots \cap V_k$ \\
$V_i \ldots $ connected components of G





%\include{02_the_next_chapter}

\printindex

\end{document}
