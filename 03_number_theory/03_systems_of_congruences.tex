
\section{Systems of Congruences}

\Theorem.
$m = m_1 m_2$, $gcd(m_1, m_2) = 1$. Then $x \equiv y (m) \iff \begin{cases} x \equiv y (m_1) \\ x \equiv y (m_2) \end{cases}$

\ProofForward.
\[
  x \equiv y (m) \iff x = y + km \implies x = y + k m_1 m_2 \implies
  \begin{cases}
    x \equiv y (m_1) \\
    x \equiv y (m_2)
  \end{cases}
\]

\ProofBackward.
\[
  \begin{cases}
    x \equiv y (m_1) \\
    x \equiv y (m_2)
  \end{cases}
\]
\begin{align*}
  & \implies \exists l: x-y = l\cdot m_1 \equiv 0 (m_2)\\
  & l \equiv 0 (m_2) \implies l = l' \cdot m_2
    \implies x-y = l'\cdot m_1 m_2
    \implies x \equiv y (m)
\end{align*}

\textbf{Corollary.}
\[
  m = \prod_{i=1}^r m_i,\; gcd(m_i, m_j) = 1 \quad \forall i \neq j
\]

Then $x \equiv y (m) \iff \forall i = 1, \ldots, r : x \equiv y (m_i)$

\subsection{Chinese Remainder Theorem}
We are faced with a system of congruence equations
\[
  x \equiv a_i (m_i) \quad 1 \leq i \leq r (*)
\]
$i \neq j \implies \gcd(m_i, m_j) = 1$

$\implies$ $(*)$ has a unique solution mod $\displaystyle{m = \prod_{i = 1}^r m_i}$.

and this is $\displaystyle{x \equiv \sum_{j=1}^r \frac {m}{m_j} \cdot b_j \cdot a_j (m)}$

where $b_j = \left( \frac{m}{m_j} \right) ^{-1} (m_j)$

\textbf{Remember from last time}
\lecturedate[\baselineskip]{2013-12-06}
\[
  x \equiv a_i (m_i), 1 \leq i \leq r (*)
\]
$\implies$ $(*)$ has unique solution mod $\displaystyle{m = \prod_{i = 1}^r m_i}$.

\[
  x \equiv \sum_{j=1}^r \frac {m}{m_j} \cdot b_j \cdot a_j (m) \text{ where }
  b_j = \left( \frac{m}{m_j} \right) ^{-1} \imod{m_j}
\]

\textbf{Example.}
System of two congruent relations.
\begin{align*}
  3x &\equiv 2 \pod{5} \\
  2x &\equiv 7 \pod{11}
\end{align*}

The previous relations are equivalent to
\begin{align*}
  x &\equiv 4 \pod{5} \\
  x &\equiv 9 \pod{11}
\end{align*}
\begin{minipage}[t]{0.5\textwidth}
\begin{align*}
  &m_1 = 5, m_2 = 11, m = 55 \\
  &a_1 = 4, a_2 = 9 \\
\end{align*}
\end{minipage}
\begin{minipage}[t]{0.5\textwidth}
\begin{align*}
  &b_1 \cdot 11 = 1 \pod{5}  \implies b_1 = 1 \\
  &b_2 \cdot 5 = 1  \pod{11} \implies b_2 = 9 \\
\end{align*}
\end{minipage}
\begin{align*}
  & x \equiv 11 \cdot 1 \cdot 4 + 5 \cdot9 \cdot9 \pod{55}\\
  & x \equiv 449 \pod{55} \\
  & x \equiv 9 \pod{55} \\
  & \text{Solutions: } \{\dots, -46, 9, 64, 119, \dots \}
\end{align*}

\Proof.

\begin{enumerate}[1)]

  \item x is a solution: $m_i, i = 1,2, \ldots, r$, are pairwise coprime
  \[
    \implies \gcd\left(\frac{m}{m_j}, m_j\right) = 1 \implies \exists b_j
  \]
  \begin{align*}
    \frac{m}{m_j} &\equiv 0 \pod{m_i} \quad \forall i \neq j \\
    \implies \sum_{j=1}^{r}\frac{m}{m_j} b_j \cdot a_j &\equiv
      \frac{m}{m_j} \underbrace{b_i}_{\left(\frac{m}{m_j}\right)^{-1} \imod{m_i}} a_i \pod{m_i} \\
    &\equiv a_i \pod{m_i}
  \end{align*}

  \item x is unique mod m:
  \begin{align*}
    & x \equiv a_i\pod{m_i} \quad y \equiv a_i \pod{m_i} \quad \forall i = 1,\ldots, r \\
    \implies & x \equiv y \pod{m_i} \quad \forall i \\
    \implies & x \equiv y \pod{m}
  \end{align*}

\end{enumerate}


