%%% EtherPad for Discrete Mathematics VO
%%% http://www.informatik-forum.at/showthread.php?104454-Notes-2013WS-VO_01
%%% Past pads:
%%%     * 2013-10-17: committed by patrikf
%%%     * 2013-10-18: committed by patrikf
%%%     * 2013-10-24: committed by patrikf
%%%     * 2013-10-25: committed by patrikf

% Discrete Mathematics Lecture Notes 2013-11-08

\subsection{Counting sets}
\lecturedate{2013-11-08}
Finite Set $A = \{a_1, \ldots, a_n\}$, $n$-element-set or $n$-set

A \dt{counting set} is a set $A = \{1, 2, \ldots, n\}$

\begin{enumerate}
  \item number of permutations: $n!$
  \item number of $k$-subsets: ${n\choose k}$
  \item number of ordered $k$-subsets: $k! {n\choose k}$
  \item number of $k$-multisets: $b_1, \ldots, b_k \in A$, the order doesn't matter, but we are allowed to choose a element several times $\implies$ ${n +k-1 \choose k}$
\end{enumerate}

There is a mapping $f$ that maps the
\[
  \text{$k$-multiset}
  \subseteq A:
  b_1 ≤ \ldots ≤ b_k
\]
to the
\[
  \text{$k$-subset}
  \subseteq \{1,2,\ldots,n+k-1\}:
  b_1 < b_2+1 < \ldots < b_k+k-1.
\]

\begin{enumerate} 
  \setcounter{enumi}{4}
  \item number of arrangements of a multiset: $\{b_1, b_1, \ldots , b_2, b_2,\ldots, b_2, \ldots, b_m, b_m, \ldots b_m\}$ 
\end{enumerate}

$b_1$ appears $k_1$ times, ...; $n$ in total

Permutations of this multiset:
$\frac{n!}{k_1! k_2! \ldots k_m!}$

\begin{enumerate}
  \setcounter{enumi}{5}
  \item number of ordered $k$-multisets over $A$: $n^k$ \\
  (we take a fixed number of positions $k$ and for each position choose any element from $A$)
\end{enumerate}

Remember:
\begin{compactitem}
  \item $\sum_{k=0}^n {n\choose k} = 2^n$
  \item Pascal's triangle
  \item ${n\choose k} + {n\choose k+1} = {n+1 \choose k+1}$
  \item $\sum_{m=0}^n {m\choose k} = {n+1 \choose k+1}$
  \item $\sum_{k=0}^n {m+k \choose k} = {m+n+1 \choose n}$
\end{compactitem}

\[
  \forall x,y\in \mathbb{C}
  \forall n\in \mathbb{N}
  (x+y)^n = \sum_{k=0}^n {n\choose k} x^k y^{n-k}
\]

$b_1, \ldots, b_k$ \; ${n \choose k} = \frac{n(n-1) \ldots (n-k+1)}{k!}$

\textbf{Definition (Extension to complex numbers).}
Let $x\in \mathbb{C}, k\in\mathbb{N}$. Then we can define
\[
  {x\choose k} = \frac{
    x (x-1) \ldots (x-k+1)
  }{
    k!
  }
\]

\Lemma.
\[
  \forall k \in \mathbb{Z}, \forall x\in \mathbb{C}:
  {x\choose k} = {x-1 \choose k-1} + {x-1 \choose k}
\]
(with ${x \choose k} = 0$ for $k < 0$).

\Theorem. (Vandermonde's theorem).
\[
  {x+y \choose n} = \sum_{k=0}^{n} {x\choose k} {y\choose n-k},
  \forall n \in \mathbb{N}, \forall x,y \in \mathbb{C}
\]

\Proof.
Assume $x,y\in \mathbb{N}$. $X,Y$ sets: $X\cap Y=\varnothing, |X|=x, |Y|=y$.

Then the left-hand side
${x+y \choose n}$
is the number of $n$-subsets of $X\cup Y$.

Choose any $n$-subset $A\subseteq X\cup Y$.
Then $A$ can be decomposed like this:
\begin{gather*}
  A = (A\cap X)\cup (A\cap Y) \\
  |A\cap X| = k, |A\cap Y| = n-k
\end{gather*}

number of unions of shape right hand side of (*) $\left(={x+y \choose n} \right)$

\[
  \sum_{k=0}^n {x\choose k} {y\choose n-k}
\]

left hand side: 
\[
  \sum_{i\in I} p_i (x) y^i
  = \sum \widetilde p_i (x) y^i
\]

lets assume $x\in \mathbb{N}$ and is fixed, then
\[
  Q_1(y) = Q_2(y) \; forall y \in \mathbb{N} 
  \implies \forall y\in \mathbb{C}
  p_i(x) = \widetilde p_i(x) \; \forall x \in \mathbb{N}
\]

\subsection{Stirling numbers}
Assume we have a set $A = \{1,2,3, \ldots ,n\}$
, permutation $\pi\in S_n$.
$S_n$ is the symmetric group; $|S_n| = n!$

We can represent $\pi$ as follows:
\[
  \begin{pmatrix}
    1&2&3&\cdots&n \\
    \pi(1)&\pi(2)&\pi(3)&\cdots&\pi(n) \\
  \end{pmatrix}
\]

2-line representation:
\[
  \begin{pmatrix}
    1&2&3&4&5&6&7 \\
    4&6&3&1&7&5&2 \\
  \end{pmatrix}
\]
The second line is also called the word representation.

Cycle representation:
$(1 4) (2 6 5 7) (3)$
Fixpoints can be omitted, so this is the same as $(14)(2657)$.

We can do calculations with permutations.

$(12)\in S_7 \implies (12)(3)(4)(5)(6)(7)$

A transposition is a permutation of just 2 elements.

Take $(12),(13)\in S_7$. Then $(12)\circ (13) = (132)$.

Every $\pi\in S_n$ can be written as a product of cycles, even as a product of transpositions (but not unique).

e.g. $(14)(2657) = (14)(27)(25)(26)$.

Of course, the order within a cycles does not matter:
$(2652)=(5726)$.

Canonical representation: generated from 2-line representation, smallest element comes first.
If additionally you start with the cycle with the largest last element first, you could even omit parentheses.

\begin{definition}
Let
\[
  s_{n,k} = \text{number of permutations of an $n$-set $A$ which have $k$ cycles}
\]
Then $s_{n,k}$ are the \dt{Stirling numbers of the first kind}.
(Fixpoints count as a cycle as well.)
\end{definition}

\Remark.
\begin{gather*}
  s_{n,1} = (n-1)! \\
  s_{n,n-1} = {n\choose 2} \\
  s_{n,n} = 1 \\
  s_{n,0} = s_{k,0} = 0 \quad n,k ≥ 1 \\
  s_{0,0} := 1 \\
  \sum_{k=0}^{n} s_{n,k} = n! 
\end{gather*}

It doesn't matter where we start the cycle, just the order of the cycle needs to stay. $(3 2 6 5 7 1 4) = (6 5 7 1 4 3 2)$

\Theorem.
\[
  \forall n,k > 0:\;
  s_{n,k} = s_{n-1,k-1} + (n-1)s_{n-1,k}
\]

\Proof.
Take a permutation 
$\pi = (1 \ldots )(\ldots) \ldots (\ldots) \in S_n \text{, with $k$ cycles}$.

How many such permutations are there?

\begin{compactitem}
  \item If $1$ is a fixed point: $s_{n-1, k-1}$.

  \item If $1$ is not a fixed point, start with a permutation of $n-1$ elements and add the element $1$ to one of the cycles. We can insert it before any of the $n-1$ elements, thus we get
$(n-1) s_{n-1,k}$ possibilities.
\end{compactitem}

\Remark.
Signed Stirling numbers of the first kind:
\[
  c_{n,k} = (-1)^{n+k} s_{n,k}
\]

\begin{definition}
Set partitions: 
Take a set $A=\{1,2, \ldots, n\}$
and decompose it such that
\[
  A = A_1 \cup A_2 \cup \cdots\cup A_k
  \quad
  \forall i\neq j: A_i \cap A_j = \varnothing
\]
Then
\[
  S_{n,k} = \text{number of set partitions of $A$ with $k$ blocks}
\]
are the \dt{Stirling numbers of the second kind}.
\end{definition}

\textbf{Example.}
Set $A = \{1,2,3,4\}$
\begin{align*}
  \text{number of 1-partitions }= 1 &= S_{4,1}\\
  \text{number of 2-partitions }= 7 &= S_{4,2}\\
  \text{number of 3-partitions }= 6 &= S_{4,3}\\
  \text{number of 4-partitions }= 1 &= S_{4,4}
\end{align*}

\Remark.
\begin{align*}
  S_{n,1} &= S_{n,n} = 1 \\
  S_{n,2} &= 2^{n-1}-1 \\
  S_{n,n-1} &= {n\choose 2} \\
  S_{0,0} &:= 1 \\
  S_{n,0} &= S_{0,k} = 0
    \quad\forall n,k ≥ 1
\end{align*}

\Theorem.
\[
  S_{n,k} = S_{n-1,k-1} + k S_{n-1,k}
\]

\Proof.
If $\{1\}$ is a block:
\[
  S_{n-1,k-1}
\]
If $\{1\}$ is not a block, it has to be element of one of the $k$ blocks:
\[
  k S_{n-1,k}
\]

\Theorem.
$\forall x\in \mathbb{C} \forall n ≥ 0:$
\begin{gather*}
  (x)_n
    := x(x-1)(x-2) \ldots(x-n+1)
    = \sum_{k=0}^{n} (-1)^{n+k} s_{n,k} x^k \quad ((x)_0:=1)
    \\
  x^n = \sum_{k=0}^{n} S_{n,k} (x)_k
\end{gather*}

\Remark.
$V_n = \{a_0 + a_1 x + \ldots + a_n x^n \mid a_i \in \mathbb{C} \}$ \\
$\{1,x,x^2, \ldots , x^n\}, \{1,(x)_1, (x)_2, \ldots ,(x)_n\}$ are bases of $V_n$ \\
$(V_n, +, \mathbb{C})$ vector space dim $V_n = n+1$

basis change: 
\[
  (S_{m,k})_{m,k= 0, \ldots,n} 
    ((-1)^{m+k} s_{m,k})_{m,k= 0, \ldots,n}
\]

\Proof.
\begin{align*}
  (x)_n
  &= (x)_{n-1} (x-n+1) \\
  &= (x-n+1) \sum_{k=0}^{n-1} (-1)^{n-1+k} s_{n-1,k} x^k \\
  &= \sum_{k=0}^{n-1} (-1)^{n-1+k} s_{n-1,k} x^{k+1} 
    + (n-1)\sum_{k=0}^{n-1} (-1)^{n+k} s_{n-1,k} x^k \\
  &= \sum_{k=1}^n (-1)^{n+k} s_{n-1,k-1} x^k
    \quad + (n-1)\sum_{k=0}^{n-1} (-1)^{n+k} s_{n-1,k} x^k \\
  &= \sum_{k=0}^n (-1)^{n+k}
    (s_{n-1,k-1} + (n-1)s_{n-1,k}) x^k \\
  &= \sum_{k=0}^n (-1)^{n+k} s_{n,k} x^k \\
\end{align*}

