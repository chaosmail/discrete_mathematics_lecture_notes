%%% EtherPad for Discrete Mathematics VO
%%% http://www.informatik-forum.at/showthread.php?104454-Notes-2013WS-VO_01
%%% Past pads:
%%%     * 2013-10-17: committed by patrikf
%%%     * 2013-10-18: committed by patrikf
%%%     * 2013-10-24: committed by patrikf
%%%     * 2013-10-25: committed by patrikf
%%%     * 2013-11-08: committed by neroburner
%%%     * 2013-11-21: committed by neroburner
%%%     * 2013-11-22: committed by neroburner
%%%     * 2013-11-28: committed by neroburner
%%%     * 2013-11-29: committed by neroburner
%%%     * 2013-12-05: committed by neroburner
%%%     * 2013-12-06: committed by neroburner
%%%     * 2013-12-12: committed by neroburner
%%%     * 2013-12-13: committed by neroburner
%%%     * 2013-12-19: committed by neroburner

% Discrete Mathematics Lecture Notes 2013-12-20

\textbf{Remember from last time.}
\lecturedate[\baselineskip]{2013-12-20}

Fields,

Field $K$, $\Char K$, Prime field $P(K)$,

$P(K) \cong \begin{cases}$
$\mathbb{Q} \\$
$\mathbb{Z}_p, \quad p \in \mathbb{P}$
$\end{cases}$

\Corollary.
$K$ finite field. The characteristic cannot be $0$. We know that $\exists p \in \mathbb{P}, n \in \mathbb{N}^{+}: |K| = p^n$.

\Proof.
$P(K)$ finite, this means $\exists p \in \mathbb{P}$ such that $|P(K)| = p, P(K) \cong \mathbb{Z}_p$

$P(K) \subseteq K$: regard $K$ as vector space over $P(K)$. Scalars are taken from $P(K)$. We have finite spaces
\begin{align*}
  \implies & \exists \text{ base } \{ a_1, a_2, \ldots, a_n \} \subseteq K, \text{i.e.} \dim \GroupGenBy{K , P(K)} = n\\
  \implies & K= \left\{\sum_{i=1}^n \lambda_i a_i \mid \lambda_i \in \underbrace{P(K)}_{p\text{ elements}}, i = 1,2,\ldots,n\right\} \\
  \implies & |K| = p^n
\end{align*}

\Remark.
In fact, given $p,n \implies $

1) $\exists K: |K| = p^n$\\
2) $K, K'$ fields$: |K| = |K'| = p^n \implies K \cong K'$

$K$ field $\implies K[x]$ Euclidean ring $\implies$ every ideal is a principal ideal. Furthermore $K[x]$ is a factorial ring (unique factorization into primes).
\begin{align*}
  & P(x)= x^n + a_{n-1} x^{n-1} + \ldots + a_1 x_1 + a_0 \qquad a_0, a_1, \ldots ,a_{n-1} \in K \\
  & I = (P(x)) \text{ ideal in $K[x]$ } \implies P(x) \in I, Q(x)P(x) \in Q(x) \cdot I \subseteq I \\
  & \{R(x) \mid \exists Q(x) \in K[x] : P(x)Q(x) = R(x)\} = I
\end{align*}

$\underbrace{K[x]/ P(x)}_{\text{quotient ring}}$ ( i.e. $K[x] / (P(x))$ ): $A(x) \equiv B(x) \mod P(x) :\Leftrightarrow P(x) ∣ A(x) - B(x)$ $\implies \equiv $ is a congruence relation.
\begin{align*}
  A(x) \equiv B(x), C(x) \equiv D(X) \mod P(x) \\
  A(x) + C(x) \equiv B(x) + D(x) \mod P(x) \\
  A(x)C(x) \equiv B(x) D(x) \mod P(x)
\end{align*}

This implies, that we have a quotient ring $K[x]/ P(x)$

$P(x) \equiv 0 \mod P(x)$

$\implies x^n \equiv -a_{n-1}x^{n-1} - a_{n-2}x^{n-2} - \ldots - a_1 x - a_0 \mod P(x)$\\
each polynomial $Q(x)$ fulfills:
$Q(x) \equiv \tilde{Q}(x) \mod P(x), \deg \tilde{Q}(x)<n$

\Example.
$\mathbb{R}[x] / x^2-1$:
\begin{leftbar}
  Remark: $x^2 \equiv 1 \mod {x² -1}$
\end{leftbar}\vspace{-1cm}
\begin{align*}
  \underbrace{x⁴}_{\equiv x²}
  \underbrace{-3x³}_{\equiv -3x}
  \underbrace{+2x²}_{\equiv 2}
  -5x + 1                       & \equiv x² - 3x +2 - 5x + 1\\
                                & \equiv \underbrace{x²}_{\equiv 1} - 8x + 3\\
                                & \equiv -8x + 4\\
\end{align*}
$\mathbb{R}[x] / x^2 -1 = \{ \overline{ax+b} \mid a,b \in \mathbb{R} \}$

$\overline{x-1}, \overline{x+1}$ are zero-divisors $\implies \mathbb{R}[x] / x^2 -1$ is not an integral domain

In general: $K[x] / P(x) = \left\{ \overline{\sum_{i=0}^{n-1} b_i x^i} \mid b_i \in K\right\}$

If $P(x) = Q(x)R(x), \deg Q(x) \geq 1, \deg R(x) \geq 1 \implies K[x] / P(x)$ is not an integral domain.

\Theorem.
$K$ field, $P(x) \in K[x] \implies K[x] / P(x)$ is a field if and only if P(x) is irreducible.

\ProofForward.
Assume $K[x] / P(x)$ is a field. Then the polynomial $P(x)$ must be irreducible, otherwise $K[x] / P(x)$ has zero-divisors.

\ProofBackward. Assume $P(x)$ is irreducible:
$K[x] / P(x)$ is a commutative ring with 1-element $\bar{1} = 1+ (P(x))$

We have to show that every non-zero element has an inverse.

$A(x) \not\equiv 0 \mod P(x) \implies $ without loss of generality $\deg A(x) < \deg P(x) \implies \gcd(A(x), P(x)) = 1$\\
$\implies \exists B(x), C(x)$ such that
\begin{align*}
  & 1 = A(x)B(x) + P(x)C(x)\\
  \implies & 1 \equiv A(x) B(x) \mod P(x) \\
  \implies & B(x) = A(x)^{-1} \text{ in } K[x] / P(x) \\
  \implies & (K[x] / P(x))^{*} = (K[x]/P(x)) \backslash \{\bar{0}\} \\
  \implies & K[x] / P(x) \text{ is a field}
\end{align*}

Remember: $(K[x] / P(x))^{*}$ is the set of units in the ring

\Remark.
\begin{enumerate}[1)]
  \item $P(x)$ is irreducible, $\deg P \geq 2$\\
  $\implies P(x)$ has no zeros, since otherwise, say $P(a) = 0$, then $x - a ∣ P(x)$, which is a contradiction.

  \item $K$ subfield of $K[x]/P(x)$\\
  $K\hat{=}$ constant polynomials
\end{enumerate}

\subsubsection{Algebraic Extensions of a Field $k$}
$\mathbb{R}, x^2 +1 = 0, x = \pm \sqrt{-1}$ \\
$\implies \mathbb{C} = \{a+bi \mid a,b \in \mathbb{R} \}$

$P(x)$ irreducible over $K$. $P(x) = 0$ has no solutions in $K$

$P(a) = 0 \implies a\notin K$, $ a \in L \supsetneqq K$

$P(x)$ is monic, i.e. $P(x) = x^n + a_{n-1}x^{n-1} + \ldots + a_1 x + a_0$ (the first coefficient is 1)

\Theorem.
If $K,L$ fields, $K\subseteq L$, $a$ is zero of some polynomial in $K[x]$

$a \notin K \implies \exists!$ monic and irreducible polynomial in $K[x]$ having $a$ as zero.

\Proof.
Existence: $K[x]$ is a factorial ring and an integral domain.

Uniqueness: Let us assume we have two polynomials $P_1(x) P_2(x)$, which are monic, irreducible, $P_1(x) \neq P_2(x), P_1(a) = P_2(a) = 0$

\begin{align*}
  \implies d(x) &= gcd(P_1(x), P_2(x))\\
                &= A(x) P_1(x) + B(x) P_2(x) \\
  \implies d(a) &= 0
\end{align*}

But we defined $d(x) = 1$, Contradiction!

$\implies P(x)$ is unique, $P(x)$ has minimal degree among all $Q(x)$ with $Q(a) = 0$. \\
$\implies$ If $P(x) = x^n + \sum_{i=0}^{n-1} p_i x^i \implies \sum_{i=0}^{n-1} p_i a^i \neq 0, \underbrace{\sum_{i=0}^{n-1} c_i a^i \neq 0}_{\hat{=} \sum_{i=0}^{n-1} c_i x^i \text{ in } \underbrace{K[x]/P(x)}_{\text{field!}}}$

$\underbrace{a^n + \sum_{i=0}^{n-1} p_i a^i = 0}_{\overline{x^n + \sum_{i=0}^{n-1} p_i x^i} = \bar{0}}$

$\implies L = \{ \sum_{i=0}^{n-1} c_i a^i \mid c_i \in K\}$ is the smallest field with $a \in L$. Moreover $L \cong K[x]/P(x)$

\begin{definition}
  $P(x)$ monic, irreducible over $K$, $\deg P(x) = n$, $P(a) = 0$,
  $L = \{ \sum_{i=0}^{n-1} c_i a^i\mid c_i \in K \} \supseteq K$

  $a$ \dt{algebraic} over $K$, $P(x)$ \dt{minimal polynomial of $a$ over $K$}, $L$ \dt{algebraic extension of $K$}, $L = K(a)$, ``$K$ adjoined $a$''
\end{definition}

\Example.
$\mathbb{C} = \mathbb{R}(i) \cong \mathbb{R}[x] / x^2 +1 = \{\overline{a+bx} \mid a,b \in \mathbb{R} \}$

\begin{align*}
  (a + bx)(c+dx)
  \equiv & ac + (ad+bc)x + bd \underbrace{x^2}_{\equiv -1} \\
  \equiv & ac - bd + (ad+bc)x
\end{align*}
\begin{align*}
  (a + bi)(c+di)
  \equiv & ac + (ad+bc)i + bd i^2 \\
  \equiv & ac - bd + (ad+bc)i
\end{align*}

\Example.
$\mathbb{Q}[x] / x^2 -2 \cong \mathbb{Q}(\sqrt{2}) = \{ a+b \sqrt{2} \mid a,b \in \mathbb{Q}\}$ \\
$x^2-2 = 0 \implies x = \pm \sqrt{2}$

$\sqrt{2}$ zero of the irreducible polynomial $x^2 -2 \implies x^2 = 2$

$\sqrt{2}$ irrational, algebraic

$\pi$ is not algebraic: $\nexists P(x) \in \mathbb{Q}[x]$ such that $P(\pi) = 0$. It is part of the transcendent numbers. Further examples are $\ln 2$ and $e$.

$\sqrt[n]{a}$ is algebraic: $x^n-a$

\Example.
$\underbrace{K[x]/ax+b}_{\cong K}$, $a,b \in K$, $ a\neq 0$ $\implies x = a^{-1}b$

\Example.
$\mathbb{Q}(\sqrt{2}): x^2 -3$ irreducible over $Q(\sqrt{2})$\\
$\implies \mathbb{Q}(\sqrt{2}, \sqrt{3})$

\Remark.
1) A maximal field has only irreducible polynomials of degree $1$. $\implies \nexists$ proper algebraic extensions $\rightarrow$ algebraically closed. Example: $\mathbb{C}$

2) If $K$ is a field then there exists a field $L$ such that $K \subseteq L$, $L$ algebraic closed.

3) If $\underbrace{|K| = p}_{\cong \mathbb{Z}_p}$ and $p\in \mathbb{P}$, then for every $n \in \mathbb{N}^{+}$ there is an irreducible polynomial $P(x) \in K[x]$. \\
$\implies \left| K[x]/P(x) \right| = \left| \{\overline{\sum_{i=1}^{n-1} c_i x^i} \mid c_i \in K\} \right| = p^n$

$\mathbb{Z}_p/P(x) \rightarrow$ the field of order $p^n$, Galois field $GF(p^n)$

\textbf{Proposition.}
Let $M(x)$ be the minimal polynomial of $a \in K$ and $f(x) \in K[x]$ such that $f(a) = 0$. Then $M(x)∣f(x)$. Obviously: If $g(x) = M(x)b(x) \implies g(a) = 0$

\Proof.
$f(x) = M(x) p(x) + q(x)$, $\deg q(x) < \deg M(x)$

$\underbrace{f(a)}_{=0} = 0 + q(a) \implies q(a) = 0 \implies q(x) = 0$
