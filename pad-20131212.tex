%%% EtherPad for Discrete Mathematics VO
%%% http://www.informatik-forum.at/showthread.php?104454-Notes-2013WS-VO_01
%%% Past pads:
%%%     * 2013-10-17: committed by patrikf
%%%     * 2013-10-18: committed by patrikf
%%%     * 2013-10-24: committed by patrikf
%%%     * 2013-10-25: committed by patrikf
%%%     * 2013-11-08: committed by neroburner
%%%     * 2013-11-21: committed by neroburner
%%%     * 2013-11-22: committed by neroburner
%%%     * 2013-11-28: committed by neroburner
%%%     * 2013-11-29: committed by neroburner
%%%     * 2013-12-05: committed by neroburner
%%%     * 2013-12-06: committed by neroburner

% Discrete Mathematics Lecture Notes 2013-12-12

\textbf{rembember from last time}
\lecturedate[\baselineskip]{2013-12-12}

\Proof.
for theorem $\mathbb{Z}_m^{*}$ is cyclic

\Lemma. $g$ is primitive root $\mod p$

$\mathbb{Z}_p^{*} = \{\bar{1}, \bar{2}, \ldots \overline{p-1}\}$

Then $g$ or $g+p$ is a primitive root $\mod p^2$

\Proof.
\begin{align*}
  &\GroupGenBy{g} = \mathbb{Z}_p^{*} \Leftrightarrow g^{p-1} \equiv 1 (p) \land \forall 1 < r < p-1 : g^r \not\equiv 1(p) \\
  &s : = ord_{\mathbb{Z}_{p^2}^{*}} (g) \Rightarrow g^s \equiv 1 (p^2) \Rightarrow g^s \equiv 1 (p) \Rightarrow s \geq p-1
\end{align*}

\textbf{Claim.}
\[
  g^{p-1} \not\equiv 1 (p^2) \text{ or } (g+p)^{p-1} \not\equiv 1(p^2)
\]

\textbf{Proof of Claim.}

If
\begin{align*}
  &g^{p-1} \equiv 1(p^2) \Rightarrow (g+p)^{p-1}
  &\equiv \underbrace{\underbrace{g^{p-1}}_{\equiv 1} + \underbrace{pg^{p-2}}_{\gcd(g,p) = 1 \Rightarrow pg^{p-1}\not\equiv 0 (p^2)}}_{\not\equiv 1} (p^2)
\end{align*}

If
\begin{align*}
  (g+p)^{p-1} \equiv 1(p^2) \Rightarrow \underbrace{g^{p-1}}_{\equiv 1 \Leftarrow} + \underbrace{pg^{p-2}}_{\not\equiv 0} \equiv 1 (p^2)
\end{align*}

Case 1. $g^{p-1} \not\equiv 1 (p^2)$
\begin{align*}
  &s \geq p, \quad\text{Note: } \varphi(p^2) = p(p-1) \Rightarrow s|p(p-1) \\
  &\Rightarrow s = p-l \text{ where } p-1 = k\cdot l \quad (k,l < p-1) \\
  &g^{pl} \equiv 1 (p^2) \\
  &g^{pl} = (g^p)^l \equiv g^l \equiv 1 (p) \\
  &\text{Contradiction! except for } l = p-1 \\
  &\implies s = p(p-1) \implies g \text{ primitive root } \mod p^2
\end{align*}

Case 2. $(g+p)^{p-1} \not\equiv 1 (p^2)$\\
simmilar

\Lemma.
$g$ primitive root $\mod p \quad(p \neq 2)$ $\implies$ either $g$ or $g+p$ is a primitive root $\mod p^e \quad(e \geq 2)$

\Lemma.
If $h$ is a primitive root $\mod p^e \implies$ $h$ or $h+p^e$ is a primitive root $\mod 2p^e$

\begin{definition}
  \dt{Carmichael function}
  $\lambda(m) = max_{\bar{a} \in \mathbb{Z}_m^{*}} ord_{\mathbb{Z}_m^{*}}(\bar{a})$
\end{definition}

\Remark.

a) $\lambda(m) | \varphi(m)$ \\
b) $p \in P \backslash\{2\} \implies \lambda(p^e) = \varphi(p^e) = p^{e-1}(p-1)$ \\
  $\lambda(1) := \varphi(1) = 1, \lambda(2) = \varphi(2) = 1, \lambda(4) = \varphi(4) = 2$\\
c) $\lambda(2^e) = 2^{e-2} \text{ for } e \geq 3$ \\
  $\lambda(2^e) = ord_{\mathbb{Z}_{2^e}}(5)$

\subsection{The order of elements of an abelion group $G$ with neutral element $e$}
\Theorem.
\[
  \ord(a) = r, \ord(b) = s, \gcd(r,s) = 1 \implies \ord(ab) = rs
\]

\Proof.
\begin{align*}
  &(ab)^{rs} = (\underbrace{a^r}_{e})^s (\underbrace{b^s}_{e})^r = e \implies \underbrace{\ord(ab)}_{n} |rs \\
  &e = (ab)^n = a^n b^n \implies a^n = (b^{-1})^n \implies e = a^{rn} = (b^{-1})^{nr} \quad (*)\\
  &(b^{-1})^k = e \Leftrightarrow b^k = e \stackrel{(*)}{\implies} b^{nr} = e \implies s|nr \implies s|n \\
\end{align*}

Analogously.
\begin{align*}
  r|n \implies n = rs
\end{align*}

\textbf{Corollary.}
\begin{align*}
  \ord(a^k) = \ord(a) \Leftrightarrow \gcd(k, ord(a)) = 1
\end{align*}

\Theorem.
If $ab \in G: \ord(a)=r, \ord(b) = s$ then $\exists c \in G: \ord(c) = \lcm(r,s)$

\Proof.
\begin{align*}
  &r = \prod_{i\in I} p_i^{e_i}, s = \prod_{i\in I} p_i^{f_i}, \quad  e_i, f_i \geq 0\\
  &\lcm(r,s) = \prod_{i\in I} p_i^{\max(e_i,f_i)}\\
  &\gcd(r,s) = \prod_{i\in I} p_i^{\min(e_i,f_i)}
\end{align*}

\begin{align*}
  \implies \lcm(r,s) * \gcd(r,s) = r*s\\
  I_1 = \{i \in I \mid e_i \leq f_i\}, I_2 = I \backslash I_1
  & \implies
     & r = \prod_{i\in I_1} p_i^{e_i} \prod_{i\in I_2} p_i^{e_i} \\
    && s = \prod_{i\in I_1} p_i^{f_i} \prod_{i\in I_2} p_i^{f_i} \\
\end{align*}
\begin{align*}
  &d_1 = \prod_{i\in I_1} p_i^{e_i}, d_2 = \prod_{i\in I_2} p_i^{f_i}, \quad \gcd(r,s) = d_1 d_2 \\
  &\implies \ord(a^{d_1}) = \frac{r}{\gcd(r,d_1)} = \frac{1}{d_1} \text{ and } \ord(b^{d_2}) = \frac{s}{d_2}
\end{align*}

Observe.
\begin{align*}
  \gcd(\frac{r}{d_1}, \frac{2}{d_2} ) = 1
    \implies \ord(a^{d_1} b^{d_2}) =
    \frac{rs}{d_1 d_2} = \lcm(r,s)
\end{align*}

\textbf{Corollary.}
If $a_1, \ldots, a_r$ with $\ord(a_i) = k_i$ then
\[
  \exists a \in G: \ord(a) = \lcm(k_1, k_2, \ldots, k_r)
\]


\subsection{And now to something completely different!}
\begin{align*}
  \lambda(n), n = \prod_{i = 1}^{r} p_i^{e_i}, e_i > 0
\end{align*}

\begin{align*}
\text{Let } &a_i \in \mathbb{Z}_{p_i^{e_i}}^{*} \text{ s.t. } \ord(a_i) = \varphi(p_i^{e_i}) = \lambda(p_i^{e_i})\\
  &b_i \in \mathbb{Z}_n \text{ s.t. } b_i \equiv 1 (p_j^{e_j}) \quad \forall j \neq i \text{ and } b_i \equiv a_i (p_i^{e_i})
\end{align*}
$b_i$ exists by Chinese remainder theorem

\textbf{Claim.}
\[
  \ord_{\mathbb{Z}_n^{*}}(b_i) = \ord_{\mathbb{Z}_{p_i^{e_i}}^{*}} (b_i) = \lambda(p_i^{e_i})
\]

\Proof.
For $b_i^k \equiv 1 (n)$ we must have $b_i^k \equiv 1 (p_j^{e_j}) \quad \forall j = 1, \ldots, r$

\begin{align*}
  &b_i = c\cdot n+1 = c c_1 p_j^{e_j} +1 \implies b_i^k \equiv a_i^k \equiv 1 (p_i ^{e_i}) \\
  &k_{\min} = \lambda(p_i^{e_i}) \implies \ord_{\mathbb{Z}_n^{*}}(b_i) \geq \lambda(p_i^{e_i}) \\
  &b_i^{k_{\min}} \equiv 1 (p_j^{e_j}) \quad \forall j \implies b_i^{k_{\min}} \equiv 1 (n)
\end{align*}

We found $b_1, b_2, \ldots, b_r \in \mathbb{Z}_n$ such that $\ord_{\mathbb{Z}_n^{*}}(b_i) = \lambda(p_i^{e_i}) \quad \forall i = 1, \ldots, r$

\begin{align*}
  &\implies \exists b \in \mathbb{Z}_n^{*} \text{ s.t. } \ord_{\mathbb{Z}_n^{*}}(b) = \lcm(\lambda(p_1^{e_1}), \lambda(p_2^{e_2}), \ldots, \lambda(p_r^{e_r})) = K \\
  &\implies \lambda(n) \geq K \text{ We want to show $\lambda(n) = K$}
\end{align*}

\Theorem.
If $ a\in G, \ord_G(a)$ maximal, then $\forall b \in G: \ord_G(b) | \ord_G(a)$

\Proof.
Exercise for the students

Let
\[
  a\in \mathbb{Z}_n: a^k \equiv 1 (n) \Leftrightarrow a^k \equiv 1 (p_i^{e_i}) \quad \forall i = 1,\ldots, r
\]

\begin{align*}
  &\lambda(p_i^{e_i}) = \text{ max order in } \mathbb{Z}_{p_i^{e_i}}^{*} \implies \ord_{\mathbb{Z}_{p_i^{e_i}}^{*}}(a) | \underbrace{\lambda(p_i^{e_i})}_{\lambda(p_i^{e_i}) | K}\\
  &\Rightarrow \ord_{\mathbb{Z}_{p_i^{e_i}}^{*}}(a) | K \implies a^K \equiv 1 (p_i^{e_i}) \quad \forall i \\
  &\Leftrightarrow a^K \equiv 1 (n) \implies \ord_{\mathbb{Z}_{n}^{*}}(a) | K \implies  \ord_{\mathbb{Z}_n^{*}}(a) \leq K \implies \lambda(n) = K
\end{align*}

\Theorem.
The Carmichael function $\lambda(n)$ obtains the following rules

$\lambda(1) = 1, \lambda(2) = 1, \lambda(4) = 2, \lambda(2^e) = 2^{e-2} $ if $e \geq 3$\\
$\lambda(p^e) = p^{e-1}(p-1)$ for $p\in \mathbb{P} \backslash\{2\}$\\
$\lambda(\prod_{i=1}^{r} p_i^{e_i}) = \lcm(\lambda(p_1^{e_1}), \ldots, \lambda(p_r^{e_r}))$

\textbf{Example.}
\[
  \lambda(100) = \lambda(2^2 \cdot 5^2) = \lcm(\lambda(4), \lambda(25)) = \lcm(2,20) = 20
\]

\textbf{Ad. RSA.}
Attack by iteration:

$n = pq$, $(n,e)$ public key, $(n,d)$ private key, \\
$x$ message, $y_0 = x, y_1 = x^e (n), y_i = y_{i-1}^{e} (n)$

Search for k: $y_k = x$

\begin{align*}
  &y_i = x^{(e^i)} (n) \\
  &ed \equiv 1 (v) \\
  &v = \lcm(p-1, q-1) = \lambda(n) \\
  &x ^{v+1} \equiv x (n)
\end{align*}

minimal $k$ s.t. $ y_k = x$ is $k_0 = \ord_{\mathbb{Z}_v^{*}}(e) = \ord_{\mathbb{Z}_{\lambda(n)}^{*}}(e)$ \\
$k_0 | \lambda(\lambda(n))$

Look for $p,q$ s.t. $\lambda(\lambda(n))$ is large.

If $p,q$ are such that $\frac{p-1}{2}, \frac{p-3}{4}, \frac{q-1}{2}, \frac{q-3}{4}$ are primes too,
then
\begin{align*}
%  \lambda(pq) = \lcm(p-1, q-1) = 2 * \frac{p-1}{2} \frac{q-1}{2} = \frac{(p-1)(q-1)}{2} \\
  \lambda(\lambda(pq)) = \lcm(\lambda(2), \lambda( \frac{p-1}{2} ), \lambda( \frac{q-1}{2} )) = \frac{(p-3)(q-3)}{8} \quad \sim \quad \frac{n}{8}
\end{align*}
